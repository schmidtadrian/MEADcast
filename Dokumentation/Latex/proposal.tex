\chapter{Proposal}

\section{Overview} % (fold)
\label{sec:Overview}
In this section I propose several modifications to the MEADcast protocol to 
overcome certain limitations.
The major modifications are:
1) Usage of Designated Routers (DRs) similar to Xcast+ \cite{xcast+} to reduce
the header size.
2) Usage of MEADcast routing tables to speed up the discovery phase via an so
called On-the-fly (OTF) discovery.

% section Overview (end)


\section{Designated Routers} % (fold)
\label{sec:Designated Routers}
The usage of DRs is derived from the Xcast+ protocol \cite{xcast+}.
In this context a slightly deviating definition for DR is used compared to the 
original one.
DRs are not required to be directly connected to the EP, rather they are just 
the closest MEADcast router towards an EP.
In addition, they are completely optional.

Due to the use of stateless routing, the MEADcast header must contain all
relevant processing information for the routers along the packets path.
This means, for each EP its address and the address of the MEADcast router
performing the M2U gets added to the destination list.
This enables a fast processing of the packets but also introduces the issue of a
large header size, which is in fact the biggest downside of MEADcast.

The size of the destination list can be significantly reduced, by solely storing
the address of each DR.
In fact this could also improve the processing time for previous MEADcast
routers, because there are less addresses to parse and lookup routes for.
This comes with the trade-off, that each DR needs to monitor which groups the 
attached EPs joined.
In Xcast+ DRs use IP multicast to monitor group membership and deliver data to 
its EPs \cite{xcast+}.
For several reasons this concept is not feasible for MEADcast: 
1) No usage of Multicast IPs
2) Clients are technology agnostic
3) EPs are not required to be directly attached to the DR, which means multicast
is may not supported by intermediate nodes.

Consequently, MEADcast DRs store group memberships as tuples of \texttt{
<<sender, channel>, EP>} in a so called Local Clients Table (LCT).
If a router receives a packet with its own address in the destination list it 
performs a LCT lookup and creates unicast packets for each EP in this group.
How EPs get added to the LCT is described in
\ref{sub:Receiving a Discovery Request}.
These additional processing costs are considered acceptable, because the
monitoring only happens close to the EPs, at the edge of the network, and can
save at least 128 Bit per EP in a data packet.
% section Designated Routers (end)


\section{Routing Table} % (fold)
\label{sec:Routing Table}
Another drawback of MEADcast is a relative long discovery phase \cite{meadcast1,
meadcast2}.
If MEADcast routers store which subnets are reachable via adjacent MEADcast
routers the duration of the discovery process could be reduced.
The first MEADcast router receiving a DRQ could immediately determine whether
the EP is reachable via MEADcast and encode this information in the DRP
addressed to the sender.
After the sender receives this DRP it can early exit the discovery phase and
start the On-the-fly discovery (See: \ref{sub:On-the-fly discovery}).

By now I am not completely sure if a speed up is really sensible, because during
the discovery phase an EP already receives data via unicast.
Well some could argue, that the concurrent delivery creates additional load on
the sender side and increases bandwidth usage.
It needs to be determined, if the speedup is worth the added complexity and
furthermore whether there are additional benefits of MEADcast routing tables
that justify their usage.


\subsection{Neighbor discovery} % (fold)
\label{sub:Neighbor discovery}
To exchange routing information MEADcast routers need to know adjacent MEADcast
routers.
The neighbor discovery can be integrated into the MEADcast discovery phase by
extending the Discovery Request (DRQ) and -Response (DRO) by a \textit{previous
hop} field.
Moreover the DRP gets send both to the sender and the address from the previous
hop field.
This ensures a two-way discovery between all adjacent MEADcast routers along a
path from sender to EP.


\subsection{Receiving a Discovery Request} % (fold)
\label{sub:Receiving a Discovery Request}

When a router receives a DRQ it performs several tasks, which can be devided in
three stages:

\paragraph{Receive} % (fold)
\label{par:Receive}
\begin{enumerate}
    \item Packet parsing:
          The router reads the previous hop field.
          If the field is empty the packet comes directly from the sender,
          though the router is its DR.
          If the field is non empty it contains the address of an adjacent
          MEADcast router, which will be added to the MCRT if its not already in
          there.
    \item Check if a matching MEADcast routing entry exists:
          If an matching entry exists, the forwarding stage can be skipped.
          The routers sets the early exit flag to true and directly enters the
          acknowledge stage.
\end{enumerate}
% paragraph Receive (end)

\paragraph{Forward} % (fold)
\label{par:Forward}
\begin{enumerate}
    \item Update DRQ: The router increments the distance field and puts its own
          address into the previous hop field.
    \item Forward DRQ: The router forwards the DRQ and waits for a response
          (DRP) until the discovery timeout is reached.
          If a response arrives in time this means, a closer so far undiscovered
          MEADcast router towards the EP was found.
          The address of the discovered router will be added to the MCRT.
          If no response arrives the router has the shortest distance towards
          the EP.
          The EPs address gets inserted into the LCT.
\end{enumerate}
% paragraph Forward (end)

\paragraph{Acknowledge} % (fold)
\label{par:Acknowledge}
\begin{enumerate}
    \item A DRP will be send to both the sender and the address
          from the previous hop field.
          If the early exit flag in the DRP is set to true, it indicates to the
          sender that an MEADcast route towards the EP is known, thus no further
          DRPs are expected and On-the-fly discovery can be used.
\end{enumerate}

% paragraph Acknowledge (end)
% subsection DRQ (end)


%\paragraph{Inbound DRQ} % (fold)
%\label{par:Inbound DRQ}
%If a router receives a DRQ it performs several tasks:
%1) Increment the distance;
%2) Read the previous hop field:
%If the field is empty the packet comes directly from the sender, though the
%router is its DR.
%If the field is non empty it contains the address of an adjacent MEADcast
%router, which will be added to the MCRT;
%3) Put own address into the previous hop field;
%4) Forward the DRQ and wait for a response (DRP) until the discovery timeout is
%reached.
%If a response arrives in time a closer MEADcast towards the EP exists.
%The address of the discovered router will be added to the routing table.
%If no response arrives the router has the shortest distance towards the EP.
%The EP gets inserted into the LCT;
%5) Send a DRP to the address from the previous hop field and the sender.
%% paragraph Inbound DRQ (end)
%% subsection Neighbor discovery (end)


\subsection{Routing Algorithm} % (fold)
\label{sub:Routing Algorithm}
After, two neighbored MEADcast routers discovered each other, they can start
to exchange routing information.
This could be done in one of the following ways.

\paragraph{Out of band} % (fold)
\label{par:Out of band}
Routers exchange routing information over some routing protocol.
An overlay OSPF could be used for that.
% paragraph Out of band (end)

\paragraph{Embedded} % (fold)
\label{par:Embedded}
Routing information could also be embedded into the DRQ and DRP.
For example, if a router receives a DRQ addressed to an EP without matching
MEADcast route, it could add its own unicast routes to the DRQ.
If there is no further MEADcast router along the path the routing informations
could be exposed, but these information shouldn't be sensible.
Similarly, a router could embed its routing infos into a DRP, if it receives a
DRQ from an unknown MEADcast router.
% paragraph Embedded (end)

Currently I favor the out of band approach because it allows the usage of any
routing algorithm without interfering with the MEADcast protocol itself.
However, the approaches are not mutually exclusive, thus could be used in
parallel.
% subsection Routing Algorithm (end)


\subsection{On-the-fly discovery} % (fold)
\label{sub:On-the-fly discovery}
The idea is to send discovery requests along with MEADcast data packets, thus
reducing the time spend in the discovery phase.
OTF discovery is applicable, if the sender receives an DRP with an early exit
flag set.

With the usage of DRs the destination list in the MEADcast header usally
solely contains router addresses. For theses the bit in the router bitmap is set
to 1 \cite{meadcast2}.
This means if the destination list contains an address with a 0 in the router
bitmap an OTF discovery is in progress.
Data packet with a new EP in the destination list gets processed as following:
\begin{enumerate}
    \item Directly connected: If the EP is directly connected, the router
          delivers the data via unicast to the EP.
          Furthermore the EP will be added to the LCT and the sender gets
          informed via a DRP.
    \item If a matching MEADcast routing entry exists, a closer MEADcast router
          is known and the packet gets forwarded as usual.
    \item If a matching unicast routing entry exists, the current router is the
          closest currently known DR and the data is delivered via unicast to
          the EP.
          The EP will be added to the LCT and the sender gets informed via a
          DRP.
          However, it could be the case that a closer so far unknown MEADcast
          router towards the EP exists.
          The router needs to verify this by sending a DRQ to the EP.
          The DRQ needs to look like it is originated from the sender, therefore
          sender, channel, distance and previous hop field have to be set
          accordingly.
          If no DRP arrives in time, the current router is the closest one.
          If a DRP arrives a closer MEADcast router towards the EP was found.
          The new DR will add the EP to its LCT and inform both the sender and
          the address from the prev-hop field (previous closest DR) via an DRP.
          After the sender receives this DRP, it needs to tell the previous DR
          that it is no longer responsible for this EP.
          Therefore the sender sends a leave message to the previous DR, which
          then removes the EP from its LCT.
          Alternatively, this could be done by removing unused entries from the
          LCT after a certain time.
\end{enumerate}
% subsection On-the-fly discovery (end)


% section Routing Table (end)


