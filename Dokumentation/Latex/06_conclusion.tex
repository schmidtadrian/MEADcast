% first fully functional (send, discovery, fallback, ...) implementation of sender and router
% comparison to uni and multicast based on various metrics
% test dynamic situations (firewall & fallback, link failure, route change)
% propose improvements
% tested own improvement (no hop by hop)
% strategic deployment at key locations can result in significant netload and upstream savings
% so far no disadvantage in using MEADcast as the sender, because of robust fallback mechanism


\chapter{Summary} % (fold)
\label{chap:Summary}
This chapter concludes this thesis.
\autoref{sec:Conclusion} provides an summary about this study, while
    \autoref{sec:Further Work} demonstrate potential avenues for future work.

\section{Conclusion} % (fold)
\label{sec:Conclusion}
% Come back to intro and goal
In the age of ever-increasing bandwidth consumption, largely driven by
    multimedia content, network operators and service providers face the
    challenge of meeting the continuously rising demands.
One potential solution is the adoption of multicast delivery.
However, the deployment of IP Multicast has fallen short of expectations.
Consequently, this research aimed to assess the efficacy of a novel multicast
    protocol known as \gls{mead}, focusing on its feasibility, performance, and
    potential application domains.

% Method / Implementation
A comprehensive evaluation of \gls{mead} was conducted by implementing its
    router functionality within the Linux Kernel alongside a standalone
    implementation of the sender.
This evaluation encompassed four distinct use cases, each representing
    real-world scenarios that potentially benefit from multicast communication.
The experimentation unfolded within a medium-sized network comprising 205
    nodes.
Furthermore, \gls{mead} was subject to comparison with established
    alternatives, notably IP Unicast and IP Multicast.

% Results / Discussion
The results highlight the feasibility of deploying \gls{mead} in medium-sized
    networks.
\gls{mead} has proven its adaptivity and suitability for dynamic network
    environments, showcasing resilience and recovery capabilities in response
    to network disruptions, particularly attributable to its well-functioning
    fallback mechanism.

Furthermore, the performance comparison emphasized significant performance
    improvements in comparison to IP Unicast.
\gls{mead} deployment yielded a 56\% reduction in network bandwidth
    utilization, an 81.23\% decrease in sender upstream bandwidth utilization,
    and a 49.18\% reduction in total transfer time (\ucii{}).
Additionally, overhead generated by the discovery phase primarily originates
    from its initial discovery, with subsequent recurring discovery phases
    incurring negligible overhead.

\gls{mead} exhibits significant improvements across all metrics for
    applications characterized by communication patterns of steady flows
    (e.g. video/audio stream) and recurring bursts (e.g. file transfer).
Furthermore, \gls{mead} has proven its efficacy across all group sizes,
    receiver distributions, and session durations.
Notably, as group sizes, receiver clustering, and session duration increase,
    the performance improvements compared to IP Unicast become more pronounced.
\gls{mead} particularly excels in scenarios demanding high throughput.
However, \gls{mead} may not be suitable for applications with very short
    session durations or those highly sensitive to initial startup latency.

\gls{mead} excels in scenarios characterized by resource constraints and
    limited network control.
Deploying a single \gls{mead} router at a strategic location can significantly
    reduce bandwidth utilization between the sender and router.
However, increasing the number of \gls{mead} routers may not necessarily
    enhance performance due to packet replication inefficiencies.
\gls{mead} empowers the sender to granularly tailor the traffic pattern
    according to specific requirements and prevailing network conditions.
With its resilient fallback mechanism and support for partial deployment,
    \gls{mead} adoption is advocated whenever multicast communication is
    applicable given the absence of major drawbacks.
Nonetheless, where IP Multicast usage is assured, it remains the preferred
    choice.

% Contribution
This thesis stands as a pioneering endeavor, providing a comprehensive
    evaluation of \gls{mead} grounded in real-world use cases, diverging from 
    preceding studies exclusively confined to network simulations and
    \glspl{sdn}.
Furthermore, our study encompasses the inaugural evaluation of \gls{mead} in its
    entirety, covering the discovery phase, data delivery phase, transition
    from IP Unicast to \gls{mead} delivery, fallback mechanism, receiver
    grouping, and anomaly handling.
Additionally, this study marks the first comparison among IP Unicast, IP
    Multicast, and \gls{mead}.
Thereby, this study has shown the feasibility of deploying \gls{mead} in
    medium-sized networks, offering promising initial insights into the
    real-world applicability of \gls{mead}.
\gls{mead} exhibits significant improvements in all metrics across
    the tested use cases, under various network conditions.
Given its resilient fallback mechanism and support for partial
    deployment, \gls{mead} adoption is advocated whenever multicast
    communication is applicable but IP Multicast is unavailable.
However, several refinements are proposed for the upcoming \gls{mead}
    revision, aiming to address existing inefficiencies and mitigate malicious
    interference.
These refinements include the omission of the Hop-by-Hop IPv6 extension header,
    integration of the ``Segments Left'' field from the static IPv6 routing
    extension header, introduction of a new ``Don't Replicate'' field,
    addressing packets to the subsequent \gls{mead} hop, and implementation of
    an authentication mechanism for discovery responses.
% section Conclusion (end)


\section{Future Work} % (fold)
\label{sec:Further Work}
% Xcast+
% P2P
% Real world test, moderately controlled environment such as an academic network.
% test over internet
% investigate grouping algorithm
This thesis illuminated several avenues for future research and development.
A natural progression from our current work involves conducting real-world
    evaluations of \gls{mead} in moderately controlled environments, such as
    academic networks.
This is particularly relevant given that our experiments were conducted in a
    virtual environment with para-virtualized network interface.
Addressing concerns regarding IPv6 extension header processing
    \cite{rfc7872_ext_hdrs_drop_rate}, further investigation could explore
    \gls{mead} transmission over the internet.

Moreover, as our study did not yield reliable results for \gls{p2p}
    communication and group sizes exceeding 100 nodes, future research
    endeavors should concentrate on addressing these aspects.
Optimizing the grouping of receivers into \gls{mead} packets has emerged as a
    complex task, warranting exploration into suitable grouping algorithms.
Additionally, since \gls{mead} relies on an out-of-band group membership
    signaling (join \& leave), implementing such mechanisms and investigating 
    the implications of highly dynamic groups require further examination.

Furthermore, the implementation and evaluation of the proposed protocol
    refinements advocated in our study should be pursued.
Additionally, investigating mechanisms to reduce the size of the address list
    is recommended.
Potential approaches could draw inspiration from \gls{xcast} extensions such
    as \gls{xcast+} \cite{xcast+}, which only store the addresses of designated
    routers in the header, resulting in significant bandwidth savings (see
    \autoref{sub:Explicit Multicast Extension}).
\gls{xcast+} stores only the addresses of designated routers in the header.
However, it is essential to consider whether these savings justify the
    introduction of state management tasks on designated routers.

% section Further Work (end)

% chapter Summary (end)
