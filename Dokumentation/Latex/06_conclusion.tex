% first fully functional (send, discovery, fallback, ...) implementation of sender and router
% comparison to uni and multicast based on various metrics
% test dynamic situations (firewall & fallback, link failure, route change)
% propose improvements
% tested own improvement (no hop by hop)
% strategic deployment at key locations can result in significant netload and upstream savings
% so far no disadvantage in using MEADcast as the sender, because of robust fallback mechanism


\chapter{Summary} % (fold)
\label{chap:Summary}

\section{Conclusion} % (fold)
\label{sec:Conclusion}
% Come back to intro and goal
In the age of ever-increasing bandwidth consumption, largely driven by
    multimedia content, network operators and service providers face the
    challenge of meeting the continuously rising demand.
One potential solution is the adoption of multicast delivery.
However, the deployment of IP Multicast has fallen short of expectations.
Consequently, this research aimed assess the efficacy of a novel multicast
    protocol called \gls{mead}, focusing on its feasibility, performance, and
    scenario identification.

% Method / Implementation
A comprehensive evaluation of \gls{mead} was conducted by implementing its
    router functionality within the Linux Kernel alongside a sender
    implementation.
This evaluation encompassed four distinct use cases, each representing
    real-world scenarios that potentially benefit from multicast communication.
The experimentation unfolded within a medium-sized network comprising 205
    nodes.
Furthermore, \gls{mead} was subject to comparison with established
    alternatives, notably IP Unicast and IP Multicast.

% Results / Discussion
The results highlight the feasibility of deploying \gls{mead} in medium-sized
    networks.
\gls{mead} has proven its adaptivity and suitability for dynamic network
    environments.
Moreover, the protocol exhibited resilience and recovery capabilities in
    response to network disruptions, particularly due to the well-functioning
    fallback mechanism.

Furthermore, the performance comparison underscores the feasibility of a
    graduate transition from extensive IP Unicast to \gls{mead} delivery,
    emphasizing significant performance improvements.
This transition yielded a 56\% reduction in network bandwidth utilization,
    an 81.23\% decrease in sender upstream bandwidth utilization, and a 49.18\%
    reduction in total transfer time (UC2) compared to IP Unicast.
Additionally, the overhead generated by the discovery phase primarily
    originates from the initial discovery, with subsequent recurring discovery
    phases incurring negligible overhead.

\gls{mead} has shown significant improvements across all metrics for
    applications characterized by communication patterns of steady flows
    (e.g. video/audio stream) and recurring bursts (e.g. file transfer),
Furthermore, \gls{mead} has proven its efficacy across all group sizes,
    receiver distributions, and session durations.
Notably, as group sizes, receiver clustering, and session duration increase,
    the performance improvements compared to IP Unicast became more pronounced.
\gls{mead} particularly excels in scenarios demanding high throughput,
    exhibiting no discernible impact on jitter, thereby rendering it suitable
    for applications sensitive to jitter fluctuations.
However, \gls{mead} may not be suitable for applications with very short
    session durations or those highly sensitive to initial startup latency.

\gls{mead} excels in scenarios characterized by resource constraints and
    limited network control.
Deploying a single \gls{mead} router at a strategic location can significantly
    reduce bandwidth utilization between the sender and router.
However, a higher number of \gls{mead} routers may not necessarily result in
    performance enhancements due to packet replication inefficiencies.
\gls{mead} empowers the sender to granularly tailor the traffic pattern
    according to specific requirements and prevailing network conditions.
With the protocol's resilient fallback mechanism and support for partial
    deployment, we advocate for \gls{mead} adoption whenever multicast
    communication is applicable, given the absence of major drawbacks.
However, in scenarios where IP Multicast usage can be ensured, IP Multicast
    remains the preferred choice.

% Contribution
This thesis stands as a pioneering endeavor in providing a comprehensive
    evaluation of \gls{mead} grounded in real-world use cases, diverging from 
    preceding studies exclusively conducted in network simulations and
    \glspl{sdn}.
Furthermore, it encompasses the inaugural evaluation of \gls{mead} in its
    entirety, covering the discovery phase, data delivery phase, transition
    from IP Unicast to \gls{mead} delivery, fallback mechanism, receiver
    grouping, and anomaly handling.
Additionally, this study marks the first comparison among IP Unicast, IP
    Multicast, and \gls{mead}.
Thereby, this study has shown the feasibility of deploying \gls{mead} in
    medium-sized networks, offering promising initial insights into the
    real-world applicability of \gls{mead}.
\gls{mead} has exhibited significant improvements in all metrics across
    all tested use cases, under various network conditions.
Given the protocol's resilient fallback mechanism and support for partial
    deployment, we advocate for \gls{mead} adoption whenever multicast
    communication is applicable but IP Multicast is unavailable.
However, we propose several refinements for the upcoming \gls{mead}
    revision, aimed at addressing existing inefficiencies and mitigating
    malicious interference.
These refinements include the omission of the Hop-by-Hop IPv6 extension header,
    integration of the ``Segments Left'' field from the static IPv6 routing
    extension header, introduction of a new ``Don't Replicate'' field,
    addressing packets to the subsequent \gls{mead} hop, and implementation of
    an authentication mechanism for discovery responses.
% section Conclusion (end)


\section{Future Work} % (fold)
\label{sec:Further Work}
% Xcast+
% P2P
% Real world test, moderately controlled environment such as an academic network.
% test over internet
% investigate grouping algorithm
This thesis indicates several avenues for future development.
A logical progression to our work is a real-world evaluation of \gls{mead} in a
    moderately controlled environment such as academic networks.
This is of particular interest, since our experiments were based on an virtual
    environment with para-virtualized network interface.
Taking into consideration concerns related to IPv6 extension header processing
    \cite{rfc7872_ext_hdrs_drop_rate}, further research could evaluate
    \gls{mead} transmission over the internet.
Moreover, since we were not able to gather reliable results for \gls{p2p}
    communication and group sizes larger than 100 nodes, future studies could
    address these aspects.
An optimal grouping of receivers into \gls{mead} packets has proven as a
    complex task, making worth the investigation of grouping algorithms.
Since, \gls{mead} is based on an out of band join and leave signaling, the
    implementation of such an mechanism and investing the implications of
    highly dynamic groups needs further investigation. 
Furthermore, the implementation and evaluation of our advocated protocol
    refinements should be considered.
Moreover, we recommend to further investigate mechanisms to reduce the size
    of the address list.
Potential approaches could draw inspiration from \gls{xcast} extensions such
    as \gls{xcast+} \cite{xcast+}.
\gls{xcast+} stores only the addresses of designated routers in the header.
The study has shown significant bandwidth savings.
However, it has to be considered whether these savings are worth the trade off
    of stateful designated routers.

% section Further Work (end)

% chapter Summary (end)
