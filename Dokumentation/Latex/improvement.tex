\chapter{MEADcast improvement}


\section{Idea} % (fold)
\label{sec:Idea}
The key idea is to reduce the header size, speedup the discovery phase and
may split groups into several data packets to reduce load.
This can be done by adapting concepts of Xcast+ \cite{xcast+} and GXcast
\cite{gxcast}.

\subsection{Grouping} % (fold)
\label{sub:Grouping}
\cite{gxcast} showed, that the fragmentation of group members into individual 
sub packets has an immense impact on the number of required packets and
therefore the performance.
Moreover it is important, how the destinations are split into sub-groups.
For example, similar to \cite{gxcast} receivers could be ordered by their IP and 
split accordingly.
% subsection Grouping (end)

\subsection{Header size} % (fold)
\label{sub:Header size}
The largest downside of MEADcast is the size of its header, more precisely its
destination list.
For each endpoint its address and if existent, also the address of its
Designated Router (DR) (Access MEADcast router) is stored.
The size of the destination list could be significantly reduced, by solely
storing the address of each DR.
In fact, also the processing time of previous MEADcast routers would be slightly
improved, because there are less addresses to parse and lookup routes for.
This comes with the trade-off, that each DR is required to monitor which
endpoint joined a certain MEADcast group, which leads to an increased processing
effort.
This should be an acceptable drawback, because it only happens close to the
receiver at the edge of the network.
This approach is derived from the Xcast+ protocol. \cite{xcast+}

With DRs an receiver-side initialization seems reasonable, but this would also
abandon agnostic clients.
We could stick with agnostic clients, if the joining/leaving of a group happens
out-of-band and the sender informs the DR about leaving endpoints.
% subsection Header size (end)
% section Idea (end)


\subsection{Routing tables} % (fold)
\label{sub:Routing tables}
Adding MEADcast routing tables to the routers could greatly speedup the 
discovery phase, by a so called ``on-the-fly discovery''.
May dynamic routing tables could provide further benefits like, (faster)
adoption to routing changes.
The MEADcast routing table contains subnets, that can be reached via other
MEADcast routers (own local subnets are not included).
% subsection Routing tables (end)


\subsection{Neighbor discovery} % (fold)
\label{sub:Routing protocol}
By adding a \texttt{prev-hop} field to the DRQ and DRP MEADcast routers are able
to gain knowledge about adjacent routers.

For example, consider the following topology: \texttt{S--M1--M2--E1}.
After an out-of-band join \textit{S} sends the following DRQ to \textit{E1}:

\begin{lstlisting}
    S  => M1             DRQ(S, 1, E1, 0, -)
    S  <= M1             DRP(S, 1, E1, 1, M1)
          M1 => M2       DRQ(S, 1, E1, 1, M1) // M2 knows M1 is neigh.
          M1 <= M2       DRP(S, 1, E1, 2, M2) // M1 knows M2 is neigh.
    S        <= M2       DRP(S, 1, E1, 2, M2) // M1 knows M2 is neigh.
                M2 => E1 DRQ(S, 1, E1, 2, M2) // will timeout
\end{lstlisting}

When \textit{M2} receives an DRQ from \textit{M1} it knows \textit{M1} is an
adjacent MEADcast router.
After \textit{M1} receives an DRP from \textit{M2} it also knows that
\textit{M2} is its neighbor.
% subsection Routing protocol (end)


\subsection{Routing protocol} % (fold)
\label{sub:Routing protocol}
After two adjacent routers know each other, they could start exchanging routing
information.
Therefore, two approaches need to be considered:

\paragraph{Separate} % (fold)
\label{par:Separate}
On the one hand, adjacent router could exchange information independent of
MEADcast (e.g. similar to OSPF).
% paragraph Separate (end)

\paragraph{Included} % (fold)
\label{par:Included}
On the other \textit{M1} could append its routing information to the forwarded
DRQ if it has no matching MEADcast routing entry for the EP.
The same applies for \textit{M2}, if it receives an DRQ from an unknown MEADcast
router it appends its routing information to the DRP.
With this approach it could be reasonable, to forward the DRP from
\texttt{prev-hop} to \texttt{prev-hop} until it reaches the sender, to
distribute routing information.\\
% paragraph Included (end)

On the first glance, the separate approach seems more promising, because it is 
the more modular approach and doesn't seem to have any disadvantages so far.

% subsection Routing protocol (end)


\subsection{On the fly discovery} % (fold)
\label{sub:On the fly discovery}
If a sender can ensure it is connected to a MEADcast router\footnotemark\ the
discovery phase can be significantly improved.
In this case we could perform the discovery within MEADcast data packets.

Following the approach of Xcast+ \cite{xcast+} the destination list solely
contains DRs, which are marked in the router bitmap \cite{meadcast2}.
If the destination list contains an address which is not marked as a router it 
is part of an on the fly discovery.
To ensure what to do with the on the fly discovery MEADcast routing tables are
required (see \ref{subsub:newendpoint}).
This table contains all subnets, a closer MEADcast router is known for.

 \footnotetext{
    May it's sensible, that the sender itself implements the DR or is able to 
    retrieve MEADcast routing infos from it.
}
% subsection Discovery Phase (end)


\section{Messages} % (fold)
\label{sec:Messages}

\subsection{Management} % (fold)
\label{sec:Management}

\paragraph{Discovery Request (DRQ)} % (fold)
\label{par:Discovery Request}
\texttt{(src, channel, dst, dist, prev-hop)}\\
Similar to MEADcast \cite{meadcast2}, added \texttt{channel} and \texttt{prev-hop}.
% paragraph Discovery Request (DRQ) (end)

\paragraph{Discovery Response (DRP)} % (fold)
\label{par:Discovery Response}
\texttt{(src, channel, dst, dist, prev-hop)}\\
Similar to MEADcast \cite{meadcast2}, added \texttt{channel} and \texttt{prev-hop}.
% paragraph Discovery Response (DRP) (end)

\paragraph{Leave Group} % (fold)
\label{par:Leave Group}
\texttt{(src, channel, DR, EP)}\\
Sender tells an Designated Router (DR), that the endpoint (EP) left the group.
Required, because of agnostic clients end DRs.
% paragraph Leave Group (end)


\subsection{Data} % (fold)
\label{sub:Data}
Data: same as MEADcast \cite{meadcast2}.
% subsection Data (end)


\subsection{Example: Router behavior} % (fold)
\label{sub:Example: Router behavior}

\subsubsection{Receives Discovery Request (DRQ)}
\begin{enumerate}\itemsep0em
    \item Increment distance
    \item Read \texttt{prev-hop} field\\
        empty \textrightarrow\ it comes from sender\\
        else  \textrightarrow\ its an adjacent MEADcast router (add it to
        MEADcast routing table)
    \item Put itself into \texttt{prev-hop} header
    \item Forward packet (DRQ) and wait for DRP till timeout is reached:\\
        DRP \textrightarrow\ a closer MEADcast router exists (add it to MEADcast
        routing table)\\
        timeout \textrightarrow\ current router is closest
    \item Send DRP to \texttt{prev-hop} and sender.
\end{enumerate}

\subsubsection{Receives Discovery Response (DRP)}
\begin{enumerate}\itemsep0em
    \item Read \texttt{prev-hop} field\\
        empty \textrightarrow\ error\\
        else  \textrightarrow\ its an adjacent MEADcast router (add it to
        MEADcast routing table)
\end{enumerate}


\subsubsection{Receives data packet with new endpoint}
\label{subsub:newendpoint}
\begin{enumerate}\itemsep0em
    \item If the endpoint is directly connected, the current router is EPs DR.
        Send a DRP to sender.
    \item If a matching MEADcast routing entry exists, forward packet.
        This indicates, its a join message for another MEADcast router.
    \item If a matching unicast routing entry exists:
    \begin{itemize}\itemsep0em
        \item Add endpoint to local group members
        \item Deliver data via unicast to endpoint
        \item Send DRP to sender to tell him, that the current router is DR for 
            the EP.
        \item Send a DRQ to the endpoint, to verify, if a closer unknown
            MEADcast router exists.
            If we get a response a closer router exists (it will tell the sender
            about it). After the sender receives the DRP from the new DR will
            tell the current router, to remove the endpoint from the local group
            members (via leave msg).
            On a timeout, the current router is the closest.
    \end{itemize}
\end{enumerate}
% subsection Example: Router behavior (end)

% section Messages (end)
