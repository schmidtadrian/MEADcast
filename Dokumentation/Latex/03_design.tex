% 1) Overarching Questions  What I want to know?
% 2) Measurements (MEADcast specific) How can I measure that
% 3) Scenarios
%    3.1) Potential Application Domains
%    3.2) Application Characteristics
% 4) Experiments
% 5) Parameters
\chapter{Experiment Design} % (fold)
\label{chap:Design}
This chapter delves into the design of a series of use cases to evaluate
    \gls{mead}.
To ensure, an comprehensive examination of \gls{mead} across a diverse range of
    use cases \autoref{sec:Application Domains} initially elaborates on
    application domains that may benefit from multicast communication.
Subsequently, \autoref{sec:Application Characteristics} discusses the specific
    composition of characteristics that distinguish these domains.
To maintain the quality of our study, \autoref{sec:Measurements} discusses
    research questions and measurements derived from the previously elaborated
    application domains and characteristics, guiding our research process.
Following this, a selection of specific use cases is presented in
    \autoref{sec:Use cases}, aimed at addressing our research
    questions.
Next \autoref{sec:Testing parameters} introduces several testing parameters
    shared across all use cases.
Finally, \autoref{sec:Requirement_Analysis} elaborates on the requirements for our
    implementation, specifically regarding the topology, sender, and router
    software.

\section{Application Domains} % (fold)
\label{sec:Application Domains}
As highlighted in the \autoref{sec:Motivation}, numerous motivations support
    the adoption of multicast communication.
Additionally, various types of applications have the potential to benefit
    from multicast communication.
% Furthermore, there are various kinds of applications, which potentially benefit
%     from Multicast.
% However, as outlined in \autoref{tab:mccomm} multicast is specifically tailored
%     for a particular form of communication -- \textit{simultaneous}
%     transmission of \textit{identical} data to multiple recipients.
However, it is essential to acknowledge that multicast is tailored for a
    specific form of communication -- \textit{simultaneous} transmission of 
    \textit{identical} data to multiple recipients, as delineated in
    \autoref{tab:mccomm}.
This inherent characteristic finds an ideal match in TV broadcasts,
    establishing them as a prime candidate for multicast communication.
Conversely, Video on Demand services such as YouTube present a less favorable
    scenario for multicast adoption, as users request videos at different
    points in time.
The advantages of multicast become particularly evident in bandwidth-intense
    applications, where its potential to significantly reduce the total
    communication volume is most pronounced.
Real-time multimedia services therefore emerge as a particularly well-suited
    domain for the application of multicast.
In the following, we introduce the formulated application domains, illustrating
    them with specific examples as detailed in \autoref{tab:mcappdom}.
The characteristics distinguishing these domains are discussed in the
    following section.

\begin{table}
    \centering
    \begin{tabular}{ccc}
    \toprule
        & \multicolumn{2}{c}{\textbf{Temporal}} \\
        \cmidrule{2-3}
        \textbf{Content} & Synchronous & Asynchronous \\
    \midrule
        Identical & TV broadcast & Video on demand \\
                  & (yes)        & (no) \\
        \addlinespace
        Different & -            & Web browsing \\
                  & (no)         & (no) \\
        % Identical & yes & no \\
        % Different & no & no \\
    \bottomrule
    \end{tabular}
    \caption{Suitability of communication patterns for Multicast}
    \label{tab:mccomm}
\end{table}

The first application domain is \textit{Multimedia Streaming}, encompassing 
    various applications transmitting multimedia content to an arbitrary number
    of destinations, such as IPTV \cite{meadcast2, ratnasamy2006revisiting},
    Internet Radio \cite{meadcast1}, podcasts, and live streaming (e.g.
    Twitch\footnote{\url{https://www.twitch.tv/}}).
\textit{Conferencing and Collaboration} is the second domain, comprising Audio
    and Video Conferences \cite{overlay_mc_routing, meadcast2,
    mc_routing_multimedia}, \gls{voip} \cite{gxcast, xcast_rfc}, as well as
    real-time collaboration applications \cite{diot2000deployment, xcast_rfc}
    like online Mind Maps and Whiteboards.
The next domain \textit{File Transfer}, covers numerous applications 
    distributing files to multiple recipients, including Software Distribution
    and Updates, Patch Management \cite{meadcast1, ratnasamy2006revisiting}, 
    Logging \cite{diot2000deployment} as well as file sharing and
    synchronization \cite{overlay_mc_routing}.
\textit{Information delivery} comprises applications pushing information to 
    multiple destinations.
One example is a news application, which notifies its users about new
    information of topics or channels they have subscribed
    \cite{diot2000deployment}.
Other applications include widely utilized smartphone notifications, RSS feeds
    \cite{ratnasamy2006revisiting}, Logging (e.g. SNMP), and Stock quotes
    \cite{cisco_ipmc}.
The last domain is \textit{Distributed Simulation}, with exemplary applications
    such physics simulations \cite{diot2000deployment}, virtual reality, and
    online gaming \cite{ratnasamy2006revisiting}.

% \textit{Multimedia streaming} presents the first application domain.
% This domain comprises various applications transmitting multimedia content to 
%     an arbitrary number of destinations, such as IPTV \cite{meadcast2,
%     ratnasamy2006revisiting}, Internet Radio \cite{meadcast1},
%     podcasts, and live streaming (e.g. Twitch).
% The next application domain is \textit{Conferencing and Collaboration},
%     encompassing Audio- and Video-Conferences \cite{overlay_mc_routing,
%     meadcast2, mc_routing_multimedia}, \gls{voip} \cite{gxcast, xcast_rfc},
%     as well as real-time collaboration tools \cite{diot2000deployment,
%     xcast_rfc} like Mind Maps and Whiteboards.
% Another domain is \textit{File Transfer}.
% This includes all forms of distributing files to multiple recipients.
% Prominent examples are Software-Distribution and -Updates and Patch Management
%     \cite{meadcast1, ratnasamy2006revisiting}, Logging
%     \cite{diot2000deployment} as well as file-sharing and -synchronization
%     \cite{overlay_mc_routing}.
% The next application domain is \textit{Information delivery}.
% This comprises information that gets pushed to multiple destinations.
% For example, in a news application users could subscribe to certain topics or
%     channels they want to receive updates from \cite{diot2000deployment}.
% Further applications are widely used smartphone notifications, RSS feeds
%     \cite{ratnasamy2006revisiting}, Logging (e.g. SNMP), and Stock quotes
%     \cite{cisco_ipmc}.
% The last domain is called \textit{Distributed simulation}.
% Exemplary applications are physics simulations, virtual reality
%     \cite{diot2000deployment}, and online gaming \cite{ratnasamy2006revisiting}.



%\cite{overlay_mc_routing}:
%   - video conference,
%   - video on demand
%   - distributed simulation (incl. online gaming)
%   - p2p file sharing

%\cite{mc_routing_multimedia}
%   - audio and video

%\cite{meadcast1, meadcast2}
%   - Audio and Video Conferencing
%   - IPTV, Internet Radio
%   - Sofware updates
%   - large scale Configuration

%\cite{diot2000deployment}
%    - real time audio and video
%    - Push applications
%    - Conferencing and Collaboration
%    - File tranfser (cache, logging)

%\cite{ratnasamy2006revisiting}
%    - Online Games
%    - IPTV
%    - File sharing
%    - Software Updates
%    - RSS
%    - Conferences

%\cite{gxcast}
%    - VoIP
%    - Video conferences
%    - Dist. interactive simulation
%    - Software upgrade

%\cite{xcast_rfc}
%    - IP telephony
%    - videoconferencing
%    - multi-player games
%    - collaborative e-meetings
\begin{table}[h]
    \centering
    \begin{tabularx}{\textwidth}{>{\hsize=.4\hsize}XX}
    \toprule
        \textbf{Domain} & \textbf{Applications} \\
    \midrule
        Multimedia streaming &
        IPTV, Internet Radio, podcasts, streaming platforms
        \\\addlinespace
        Conferencing \& Collaboration &
        Audio- \& Video-Conferences, \gls{voip}, Mindmaps, Whiteboards, \dots
        \\\addlinespace
        File Transfer &
        Software Distribution, Updates, Patch Management, File-sharing and
            -synchronization, Logging
        \\\addlinespace
        Push notifications &
        RSS-feed, Logging, Stock quotes
        \\\addlinespace
        Distributed simulation &
        Online Gaming, Virtual World, Simulation \\

    \bottomrule
    \end{tabularx}
    \caption{Multicast application domains}
    \label{tab:mcappdom}
\end{table}
% section Application Domains (end)

\section{Application Characteristics} % (fold)
\label{sec:Application Characteristics}
This section delineates various characteristics derived from the application
    domains, categorized into three groups, as detailed in
    \autoref{tab:appscenarios}.
\paragraph{Group} % (fold)
\label{par:Group}
% - size
% - duration
% - membership (static vs. dynamic)
% - distribution
Group communication occurs in diverse forms.
The number of participants in a multicast group displays substantial variation,
    ranging from small ($<10$), to medium-sized and up to large-sized ($>100$)
    groups.
Additionally, the session duration exhibits significant diversity, spanning
    from a few minutes up to several days.
The group \textit{membership} can either be static or dynamic.
For instance, small to medium-sized conferences might last several minutes
    to a few hours with mostly static group membership.
In contrast, broadcasts of music or sports events endure for several days,
    attracting millions of viewers who may join or leave at any time.

\paragraph{Communication} % (fold)
\label{par:Communication}
% - symmetric vs. asymmetric (m:n vs. 1:n)
% - Steady stream vs. bursts
The communication \textit{pattern} within a group can either be symmetric
    (m:n) or asymmetric (1:n).
Furthermore, the \textit{interval} during which participants exchange messages
    may constitute a steady flow (e.g. video stream) or a recurring burst (e.g.
    file transfer).
In a \gls{p2p} video conference or online game, all group members continuously
    transmit data to each other.
Conversely, in an RSS feed, a single server periodically pushes information to
    multiple subscribers.
% paragraph Communication (end)

\paragraph{Network} % (fold)
\label{par:Network}
% - Throughput
% - Latency and Jitter
% - Drop rate
The network requirements of different applications exhibit significant
    variation.
Specific applications exhibit sensitivity to distinct factors, with some
    prioritizing low \textit{latency} and \textit{jitter}, others demanding
    high \textit{throughput}, and yet others being sensitive to the packet
    \textit{drop rate}.
For instance, many online games rely on low latency while maintaining frugal
    bandwidth requirements \cite{games_net_req}.
In contrast, ensuring a high-quality video stream necessitates elevated
    throughput, even though increased latency is acceptable due to prevalent
    client-side buffering.
Furthermore, an RSS feed has low throughput and moderate latency requirements.
Nevertheless, it is sensitive to the packet drop rate, since it has to ensure
    successful data delivery.
% paragraph Network (end)
% subsection Application Characteristics (end)
% paragraph Application Domains (end)

\begin{table}
    \centering
    \begin{threeparttable}
    \begin{tabular}{lcccccccc}
    \toprule
        & \multicolumn{3}{c}{\textbf{Group/Session}}
        & \multicolumn{2}{c}{\textbf{Communication}}
        & \multicolumn{3}{c}{\textbf{Network}} \\
        \cmidrule(lr){2-4}\cmidrule(lr){5-6}\cmidrule(lr){7-9}
        \textbf{Domain}
        & \makecell{Dura-\\tion\tnote{1}} & \makecell{Mem.\\ship} & Size\tnote{2}
        & \makecell{Pattern} & \makecell{Interval}
        & \makecell{Pkt.\\size} & \makecell{Through.\\(tx/rx)} & \makecell{Latency\\/Jitter} \\
    \midrule
        % Livestream      & h-d   & dyn.  & l     & 1:n   & steady    & med.  & high/med. & med.  \\
        % Conference      & m-h   & stat. & s-m   & m:n   & steady    & med.  & high/high & low   \\
        % File transfer   & s-h   & stat. & s-l   & 1:n   & burst     & large & high/high & high  \\
        % Push info.      & s     & stat. & m-l   & 1:n   & burst     & small & med./low  & med.  \\
        % Online game     & m-h   & stat. & s-m   & m:n   & steady    & small & low/med.  & low   \\
        Multimedia      & h-d   & dyn.  & l     & 1:n   & steady    & med.  & high/med. & med.  \\
        Conference      & m-h   & stat. & s-m   & m:n   & steady    & med.  & high/high & low   \\
        File transfer   & s-h   & stat. & s-l   & 1:n   & burst     & large & high/high & high  \\
        Push info.      & s     & stat. & m-l   & 1:n   & burst     & small & med./low  & med.  \\
        Dist. sim.     & m-h   & stat. & s-m   & m:n   & steady    & small & low/med.  & low   \\
    \bottomrule
        
    \end{tabular}
    \begin{tablenotes}
    \item [1] \textsl{(s)econds, (h)ours, (d)ays}
    \item [2] \textsl{(s)mall, (m)edium, (l)arge}
    \end{tablenotes}
    \end{threeparttable}
    \caption[Characteristics of the application domains]{Characteristics of the application domains (based on \autocite{cartesian_us_bw, diot2000deployment})}
    \label{tab:appscenarios}
\end{table}
% section Application Characteristics (end)

% Feasibility:
% - How robust is the current MEADcast specification?
% - Are there any issues or opportunities for improvement with the current 
%   protocol specification

% Performance:
% - How does MEADcast perfom compared to unicast and IP Multicast
% - How big is the overhead of the discovery phase?
% - How does the degree of deployed MEADcast routers impact the results?

% Scenarios:
% - Under which conditions is the usage of MEADcast sensible?
% - Which applications and characteristics are well served by MEADcast?


\section{Measurements} % (fold)
\label{sec:Measurements}
% Based on the thesis goal we first formulate overarching research questions we
%     endeavor to answer with our experiments. 
% Therefore, these questions guide our design process for the experiments.
% Following, this section depicts how we answer these questions.
Aligned with the goal of this thesis, this section formulates overarching
    research questions derived from the previously introduced potential
    application domains, guiding the design of the use cases.
Additionally, this section outlines our approach to answering these questions.

\begin{enumerate}
    \item[\textit{RQ1}]\label{rq1}
        \textit{How robust is the current \gls{mead} specification?}\par
        While prior research on \gls{mead} has primarily taken place in simulated
            environments \cite{meadcast1, meadcast2} and small stub networks
            \cite{sdn_ba}, this thesis endeavors to assess \gls{mead} in a more
            realistic setting.
        To achieve this, we perform a stress test to evaluate the protocol's
            behavior in a less ``clinical'' environment.
        During this test we observe \gls{mead}'s reaction to deliberate routing
            changes, to evaluate its adaptivity and suitability for dynamic
            network environments.
        Further, we simulate network disruptions by inducing router outages, to
            assess \gls{mead}'s resilience and recovery capabilities.
        The stress test also examines \gls{mead} anomaly handling by injecting
            modified discovery responses.
        Additionally, a firewall is employed to intentionally drop \gls{mead}
            packets during the data delivery phase, enabling us to evaluate the
            efficacy of the fallback mechanism.
        The results of the stress test shed light on the robustness of the
            current \gls{mead} specification, especially in diverse network
            conditions.
        This investigation aims to provide valuable insights into the
            real-world applicability and resilience of \gls{mead}.
        \item[\textit{RQ2}]\label{rq2}
        % typical metrics: throughput, latency, jitter, resource utilization,
        %   ... in different scenarios
        % impact of the discovery phase
        \textit{How does \gls{mead} perform compared to IP Unicast and
        IP Multicast?}\par
        In addition to assessing the robustness of a protocol, its performance,
            especially in comparison to existing alternatives, is a pivotal
            factor influencing its adoption.
        To address this research question, comparative performance measurements
            for \gls{mead}, IP Unicast, and IP Multicast are conducted across a
            series of use cases elaborated in \autoref{sec:Use cases}.
        These measurements encompass metrics such as throughput, latency,
            jitter, and resource utilization.
        Special attention is given to the impact of \gls{mead}'s discovery phase
            on its performance.
        This encompasses assessing the overhead produced by the discovery
            mechanism.
        Furthermore, we evaluate how the protocol's shift from extensive
            unicast to \gls{mead} delivery affects the performance metrics.
        The outcomes contribute not only to evaluating \gls{mead} performance but
            also to drawing conclusions for subsequent research questions.
        \item[\textit{RQ3}]\label{rq3}
        \textit{Which applications and characteristics are well served by
        \gls{mead}?}\par
        % first look at motivations, application domains and characteristics
        % for multicast. Then select scenarios to portray a variety
        Addressing this research question involves designing a series of 
            use cases depicting a diverse range of applications with distinct
            characteristics (see \autoref{sec:Use cases}).
        The selection of scenarios is guided by an initial exploration of
            motivations for employing multicast communication, and more
            specifically, \gls{mead}.
        These motivations lay the groundwork for identifying a range of
            potential application domains where \gls{mead} deployment could prove
            beneficial.
        Subsequently, we formulate application characteristics distinguishing
            these domains.
        Based on the gathered insights, various scenarios are chosen to
            represent different application domains and their unique set of
            characteristics.
        This comprehensive selection ensures a throughout examination of
            \gls{mead}'s suitability across a diverse spectrum of use cases.
        The results aim to delineate \gls{mead}'s application space by
            identifying its strengths and limitations.
    \item[\textit{RQ4}] \label{rq4}
        \textit{In which conditions is the usage of \gls{mead} sensible?}\par
        % testing parameters based on scenarios
        % Number of EPs
        % Number of MEADcast routers
        % EP distribution
        % Available Bandwidth
        The viability of employing \gls{mead} is influenced not only by
            application domains and their characteristics but also by
            prevailing circumstances.
        To investigate this aspect, the experiments are executed with various
            parameter configurations.
        For instance, the level of network control and the number of available
            \gls{mead} routers affect the performance and thus the viability of
            \gls{mead}.
        This applies equally to testing parameters like available bandwidth,
            number of endpoints, and their distribution.
        % Further testing parameters are the available bandwidth, the number of
        %     receivers, and their distribution.
        The goal is to provide insights into the conditions under which
            deploying \gls{mead} is advantageous compared to existing
            alternatives.
        The results aim to offer guidance for making informed decisions
            regarding the adoption of \gls{mead} in specific circumstances.
        % The results aims to provide guidance, under which circumstances
        %     employing MEADcast is favorable in comparison to existing
        %     alternatives.
\end{enumerate}


\section{Use cases} % (fold)
\label{sec:Use cases}
% Why theses scenarios?
% Real-life example?

% General goal:
% - Test as many characteristic as possible and combinations within the scope
%   of this thesis
% - Test typical real-world scenarios to assert MEADcast and propose application
%   scenarios where it excels

% Live Stream: 1:n, large, steady, med through., med latency
% - One of the most typical and promising use cases
% - Live Stream wan't to lower egress bill, aspiring platform overcoming resource limitation

% Conference: m:n, med, steady, high through., low latency
% - One of the most typical and promising use cases
% - Expending serivce offering for low bandwidth customers
% - Low Latency and m:n

% File transfer: 1:n, large, burst, high through., high latency
% - high throughput over limited time periode
% - also promising bc. latency is not important
% - Distribute files to backup cloud instances
% - Deploy an software update

Based on the previously exhibited application domains and characteristics this
    chapter formulates several use cases.

The primary goal is, to develop a series of use cases encompassing various
    domains and characteristics, aimed at answering our research questions.
This comprehensive selection ensures a throughout examination of \gls{mead}'s
    suitability across a diverse spectrum of use cases.
Therefore, the results from the use cases are pivotal to answer \textit{\rqiii{}}.
The specific characteristics of each corresponding application domain are
    illustrated in \autoref{tab:appscenarios}.
The chosen implementation for each use case is provided in
    \autoref{sub:Network Topology}.

\paragraph{UC1: Live Stream} % (fold)
\label{par:EX1: Live Stream}
The first use case involves a video live stream, representing the multimedia
    stream domain.
As reported by \citeauthor{cartesian_us_bw} \cite{cartesian_us_bw}, multimedia 
    traffic is accountable for more than half of the bandwidth consumed in
    2020.
Therefore, this experiment is indicative of a major application space, with the
    potential for significant bandwidth savings.
% Conceivable scenarios are, that a major streaming provider wants to lower their
%     egress costs, or an aspiring platform has to overcome preserving resource
%     limitations to satisfy rapidly growing bandwidth demands.
A conceivable scenario involves organizations such as universities or companies
    streaming events across multiple buildings on a campus or even across
    various locations.
Typical campus layouts encompass both \textit{local clustering} at certain
    offices or areas like the library, as well as a \textit{low receiver
    density} for remote offices, resulting in \textit{varying distances}
    between sender and receivers.
Live streams usually endure \textit{multiple hours}, attracting a \textit{large
    number} of viewers who may \textit{join or leave at any time}.
Additionally, a common requirement for HD video streams is a \textit{downstream
    bandwidth of 5 Mbit/s} \cite{cartesian_us_bw}.
This use case is well-suited to showcase \gls{mead}'s suitability for
    long-running constant data streams, highlighting the effects of different
    receiver distributions.
Given that multimedia streams generate both high traffic volume both at the
    sender and the network, it is of particular interest to explore the extend
    to which \gls{mead} can reduce the occupied sender upstream and network
    bandwidth.
Moreover, we examine whether this reduction is accompanied by trade-offs
    such as increased latency/jitter or higher CPU utilization at the sender.
% How much reduce the traffic volume on the sender
% paragraph EX1: Live Stream (end)

\paragraph{UC2: File Transfer} % (fold)
\label{par:EX2: File Transfer}
The second use case emulates a file distribution, thus representing the
    file transfer domain.
Especially scenarios such as software updates and file backups share major
    characteristics with this use case.
Our use case simulates the distribution of a software update for a server
    \gls{os}.
Servers are usually located within a single network domain, leading to high
    receiver clustering.
Additionally, updates are commonly obtained from an external source, often
    residing in another network domain.
% During the transmission, receivers fully utilize the available downstream
%     bandwidth.
A software update is characterized by the utilization of all available
    downstream bandwidth on the receivers for a limited duration.
Although mechanisms exist to limit bandwidth consumption, we only consider 
    scenarios of full downstream bandwidth utilization.
The number of receivers varies significantly, ranging from a few servers to
    entire data centers.
The primary objective of this experiment is to investigate how well \gls{mead}
    handles recurring bursts of high traffic volume.
Additionally, we aim to asses the reduction in total network bandwidth
    consumption, sender upstream bandwidth consumption, and total transfer time 
    compared to IP Unicast.
The characteristics of this use case are predestined to highlight \gls{mead}'s
    performance under ideal conditions.
% The characteristics of this experiment are predestined to measure the effects
%     of the discover phase on metrics such as bandwidth and latency.
% By adjusting parameters like communication interval and group size, this
%     experiment can be conducted analogous to \textit{EX1}.
% paragraph EX2: File tranfer (end)

\paragraph{UC3: P2P Video Conference} % (fold)
\label{par:EX3 Video conference}
The third use case simulates a \gls{p2p} video conference.
Video Conferences commonly involve small to medium-sized groups and require a
    steady communication of 1-3 Mbit/s downstream and 0.5-1 Mbit/s upstream
    bandwidth \cite{cartesian_us_bw}, lasting from several minutes to a few hours.
Our use case represents a video conference commonly  seen in modern distributed
    work culture, involving locally clustered sub-teams (e.g. office) and some
    isolated participants (e.g. customer or service provider), with varying
    distances between sender and receivers.
Furthermore, the inherent characteristic of \gls{p2p} communication facilitates
    an environment in which the effects of high \gls{mead} traffic volume on
    \gls{mead} routers can be examined.
This experiment is designed to investigate, whether \gls{mead} is capable of
    bridging asymmetric access link bandwidth \cite{xcast_rfc,cartesian_us_bw}.
Additionally, we examine the suitability of \gls{mead} for \gls{p2p}
    communication.
Lastly, we evaluate \gls{mead}'s network bandwidth and sender upstream
    bandwidth utilization for small to medium-sized groups in comparison to IP
    Unicast and IP Multicast.
% paragraph EX3 Video conference (end)

\paragraph{UC4: Online Gaming} % (fold)
\label{par:EX4 Online game}
% Latency and Jitter, requires long distances
% Test MEADcast in suboptimal setting in terms of distribution
The last use case is an online multiplayer game, representing the application
    domain distributed simulation.
Online games commonly encompass small to medium-sized groups, and require a
    steady downstream bandwidth of 0.5-1 Mbit/s \cite{cartesian_us_bw},
    enduring for several minutes to a few hours.
Most online games are highly sensitive to Latency and Jitter.
To asses \gls{mead}'s implications on these metrics, multiple \gls{mead}
    hops are essential.
Consequently, this experiment has two primary objectives.
First, to measure \gls{mead}'s effect on latency and jitter.
Secondly, to evaulate \gls{mead}'s performance in suboptimal settings,
    particularly with a high receiver distribution.
% First since, online games are highly sensitive to Latency and Jitter we examine
%     MEADcast's effect on these metrics.
% Secondly, the overhead of the protocol can be analyzed, hence online games have
%     low bandwidth requirements.
% By adjusting parameters such as packet size and communication interval, this
%     experiment can be conducted analogous to \textit{EX3}.

% paragraph EX4 Online game (end)

% Scenario      Pattern Interval    pkt size    throuput (eps)  latency
% Live Stream   1:n     steady      med         med             med
% File transfer 1:n     burst       large       high            high
% Conference    m:n     steady      med         high            low
% Online Game   m:n     steady      small       low             low
% \begin{table}[h!]
%     \centering
%     \begin{threeparttable}
%     \begin{tabular}{lcccccccc}
%     \toprule
%         & \multicolumn{2}{c}{\textbf{Communication}}
%         & \multicolumn{3}{c}{\textbf{Network}}
%         & \multicolumn{3}{c}{\textbf{Group/Session}} \\
%         \cmidrule(lr){2-3}\cmidrule(lr){4-6}\cmidrule(lr){7-9}
%         \textbf{Scenario}     & \makecell{Pattern}      & \makecell{Interval} & \makecell{Pkt.\\size} & \makecell{Through.\\(tx/rx)} & \makecell{Latency\\/Jitter} & \makecell{Dura-\\tion\tnote{1}} & \makecell{Mem.\\ship} & Size\\
%     \midrule
%         Live Stream   & 1:n     & steady      & med.         & high/med.       & med.    & h-d   & dyn.  & l\\
%         File transfer & 1:n     & burst       & large        & high/high       & high    & s-h   & stat. & s-l\\
%         Conference    & m:n     & steady      & med.         & high/high       & low     & m-h   & stat. & s-m\\
%         Online Game   & m:n     & steady      & small        & low/med.        & low     & m-h   & stat. & s-m\\
%     \bottomrule
%         
%     \end{tabular}
%     \begin{tablenotes}
%     \item [1] \textsl{(s)econds, (h)ours, (d)ays}
%     \end{tablenotes}
%     \end{threeparttable}
%     \caption{Show diff scenarios meeting}
%     \label{tab:ex_char}
% \end{table}
% section Use cases (end)

\section{Testing parameters} % (fold)
\label{sec:Testing parameters}
% - number of clients
% - distribution & clustering of clients
% - number & location of MEADcast routers
% - group size (per packet)
% - discovery interval
The series of use cases introduced earlier shares several testing parameters.
Consequently, these scenarios incorporate diverse parameter configurations
    such as the number of \gls{mead} router, intended to assess their influence on the results from our experiments, specifically addressing \textit{\rqii{}} and \textit{\rqiv{}}.

The first parameter is the \textit{number and distribution of receivers}.
Generally, with a larger number of receivers, a multicast protocol tends to
    achieve proportionally higher bandwidth savings compared to unicast.
However, we anticipate that the distribution of receivers significantly impacts
    \gls{mead}'s performance.
Therefore, the use cases encompass various distributions, ranging from highly
    clustered receivers with minimal internal distances to uniformly spread
    distributions.
This parameter aims to illuminate how spatial receiver arrangement influences
    the protocol's efficiency.

The second testing parameter is the \textit{degree of \gls{mead} support}.
Analogous to the first parameter, we hypothesize that the number and placement
    of \gls{mead}-capable routers have a significant impact on the protocol's
    performance.
To investigate this, the experiments are conducted ranging from a minimal
    degree of \gls{mead} routers to complete protocol support within the testbed.
By varying this parameter, we aim to assess \gls{mead}'s efficiency under
    conditions of limited network control, offering valuable insights
    directly addressing \textit{\rqiv{}}.

The third, parameter resolves around \textit{grouping receivers} into \gls{mead}
    packets.
Evaluating the balance between the number of recipients per packet and the
    available \gls{sdu} size is crucial.
A larger number of recipients per packet potentially reduces traffic volume by
    consolidating identical packet streams.
However, this also enlarges the \gls{mead} header, consequently diminishing the
    available \gls{sdu} size.
This trade-off might inadvertently increase traffic volume since more packets
    are required to transmit the total payload.
Furthermore, the grouping algorithm itself is considered.
For instance, recipients sharing the same parent router could be distributed
    across different packets to achieve an optimal balance between the \gls{mead}
    header and payload size.
Moreover, receivers with distinct parents might be merged under a common
    ancestor to accommodate them within a single packet.

Lastly, the effect of different values for the \textit{discovery interval} is
    also examined.
Shorter intervals facilitate quicker adaption to topology changes,
    potentially leading to enhanced receiver grouping and faster error
    recovery.
However, shorter intervals also result in higher traffic volume, which may
    impede data delivery.

% section Testing parameters (end)

\section{Requirement analysis} % (fold)
\label{sec:Requirement_Analysis}
Expanding upon on the formulated measurements and use cases, this section
    provides detailed insights into the requirements for our implementation.
\autoref{sub:Topology} discusses the requirements for the network topology
    necessary for conducting our experiment, while \autoref{sub:Software}
    examines requirements for the router and sender implementation.

\subsection{Topology} % (fold)
\label{sub:Topology}
Initially, the topology must transcend basic structures like plain bus or simple
    ring setups, ensuring a \textit{non-trivial connected graph}.
A graph is termed connected when there exists a path between every pair of
    vertices, representing clients and routers in this context.
In order to thoroughly evaluate a multicast protocol, the topology needs to
    encompass a substantial number of nodes.
Therefore, a \textit{medium-sized topology} comprising around 200 nodes
    \cite{cisco_net_size}, featuring a proportional mix of clients and routers,
    is pivotal.

\paragraph{Realistic topology} % (fold)
\label{par:Realistic topology}
Aligned with \textit{\rqi{}}, the graph should reflect a realistic medium-sized
    network topology.
As discussed in \autoref{sec:Hierachical Three-Layer Internet Model}, a
    realistic topology comprises three network layers: Core, Distribution, and
    Access Layer.
The Core Layer interconnects different network domains, facilitating high-speed
    and high-volume data transmission.
Additionally, this layer is characterized by a relatively simple design and
    should offer high availability and redundancy.
The Distribution Layer is required to aggregate traffic from multiple Access
    Layers and to provide connectivity to the rest of the network by
    transmitting packets to the core layer.
The Access Layer layer is required to facilitate network access for end
    devices, encompassing PCs and smartphones.
This layer must establish connectivity between end devices and serves as the
    demarcation point between the network infrastructure and computing devices.
% paragraph Realistic topology (end)

\paragraph{Endpoint distribution} % (fold)
\label{par:Endpoint distribution}
The use cases require various numbers of endpoints and their distribution.
The topology must accommodate group sizes ranging from small to medium-sized
    (\uciii{}, \uciv{}) up to medium to large-sized groups (\uci{}, \ucii{}).
Moreover, \uci{}, \ucii{}, and \uciii{} require areas of \textit{high receiver clustering}.
Additionally, \uci{}, \uciii{}, and \uciv{} mandate areas characterized by \textit{low
    receiver density}.
Since \uci{} and \uciii{} require both local clustering and low receiver density, the
    topology must facilitate \textit{varying distances} between sender and
    receivers.
All use cases require the integration of \textit{multiple network domains}.
\uci{}, \uciii{}, and \uciv{} employ isolated remote hosts, necessitating the incorporation
    of paths of \textit{sufficient length}.
Additionally, the comprehensive evaluation of \gls{mead}'s performance across
    diverse levels of network support and various grouping strategies, also
    requires paths of sufficient length, allowing packets to traverse multiple
    \gls{mead} routers for a thorough assessment, particularly of latency and
    jitter.
% paragraph Endpoint distribution (end)

\paragraph{Flexibility} % (fold)
\label{par:Flexibility2}
Another crucial requirement is the flexibility of the topology.
Diverse use cases and network conditions demand a modular and adaptable
    topology.
Various placements of senders, receivers, and routers are necessary.
Moreover, the use cases encompass receiver distributions ranging from highly
    clustered to uniformly spread.
Additionally, the ability to adjust the quantity of \gls{mead}-capable routers
    is crucial for evaluating various levels of \gls{mead} support.
% paragraph Flexibility (end)

\paragraph{Connectivity} % (fold)
\label{par:Connectivity}
Finally, several prerequisites pertain to the connectivity of the topology.
Given the evaluation of a Layer 3 IPv6-based protocol, the presence of multiple
    subnets and IP routing is mandatory.
Beyond that, IP Multicast routing is indispensable for a comparison of \gls{mead}
    and IP Multicast.
Furthermore, our use cases mandate various up and downstream capacities.
\uciii{} and \uciv{} require moderate downstream requirements of 1-3 Mbit/s and 0.5-1
    Mbit/s, respectively.
\uci{} and \ucii{} demand a downstream bandwidth of at least 5 Mbit/s.
Moreover, \uci{}, \ucii{}, and \uciii{} also mandate an upstream bandwidth of multiple
    Mbit/s on the sender side.
Additionally, assessing the resilience of \gls{mead} in dynamic network
    environments necessitates the inclusion of \textit{alternative paths} to
    simulate routing alterations and network disruptions.
Furthermore, to pursue \textit{\rqi{}} the incorporation of an intermediate node,
    capable dropping packets is required.
% paragraph Connectivity (end)

% paragraph Network Links (end)

% bw 5Mbit/s, 1-3 Mbit/s, 0.5-1Mbit/s
% multiple 
% subsection Topology (end)

\subsection{Software} % (fold)
\label{sub:Software}
\label{sub:Requirements}
% Based on the previously introduced experiments this section elaborates on 
%     requirements for the testbed.
% We structure the requirements into 3 categories:

% \paragraph{Topology} % (fold)
% \label{par:Topology}
% % graph
% First, the topology must transcend basic structures like plain bus or simple
%     ring setups, ensuring a \textit{non-trivial connected graph}.
% A graph is termed connected when there exists a path between every pair of
%     vertices, which in this context represents clients and routers.
% In order to thoroughly evaluate a multicast protocol, the topology needs to
%     encompass a substantial number of nodes.
% Therefore, a \textit{medium-sized topology} comprising around 200 nodes
%     \cite{cisco_net_size},
%     featuring a proportional mix of clients and routers, is pivotal.
%
% In line with \textit{RQ1}, the graph should reflect a \textit{realistic
%     network topology} of this size.
% Moreover, assessing the resilience of MEADcast in dynamic network environments
%     necessitates the inclusion of \textit{alternative paths} to simulate
%     routing alterations and network disruptions.
% Additionally, a thorough evaluation of MEADcast's performance across 
%     diverse levels of network support and various grouping strategies, requires
%     the topology to incorporate paths of \textit{sufficient length}.
% This allows packets to traverse multiple MEADcast routers, ensuring a
%     comprehensive assessment.
%
% % flexibility
% Another important requirement is the flexibility of the testbed.
% Diverse experiments and network conditions demand a modular and adaptable
%     topology.
% Various placements of senders, receivers, and routers are necessary.
% Moreover, the testbed should encompass receiver distributions ranging from
%     highly clustered to uniformly spread.
% Additionally, the ability to adjust the quantity of MEADcast-capable routers is
%     crucial for evaluating various levels of MEADcast support.
%
% % connectivity
% Finally, several prerequisites pertain to the connectivity of our testbed.
% Given the evaluation of a Layer 3 IPv6-based protocol, the presence of multiple
%     subnets and IP routing is mandatory.
% Beyond that, IP Multicast routing is indispensable for a comparison of MEADcast
%     and IP Multicast.
% Moreover, to pursue \textit{RQ1} the capability to deploy firewall mechanisms
%     is required.
% % paragraph Topology (end)

This section discusses requirements for the router and sender software
    implementation.
% Sender
% - Send discovery requests, receive discovery responses
% - Adjustable discovery interval
% - Adjustable discovery timeout
% - Adjustable grouping of receivers into MEADcast packets
% - Transimtting data both via MEADcast as well as unicast (fallback mechanism)
\paragraph{Sender} % (fold)
\label{par:Sender}
To ensure a comprehensive assessment of \gls{mead}, the sender should include the
    following features:

First, the sender must periodically dispatch discovery requests to the
    receivers.
Additionally, the sender needs to receive discovery responses and process them.
After, the discovery timeout is exceeded the sender should start grouping the
    endpoints.
Therefore, the sender is required to efficiently organize and group receivers
    into \gls{mead} packets, facilitating streamlined data transmission.
Whenever possible, the software should support \gls{mead} data transmission.
However, unicast is used as a fallback mechanism, if \gls{mead} is not available
    on certain paths.
The discovery interval, discovery timeout, and grouping functionalities should
    be configurable, allowing flexibility for experimentation and optimization
    based on various network conditions and scenarios.
Considering our evaluation of \gls{mead} is based on metrics such as latency and
    jitter, the sender should possess the capability to either measure these
    metrics directly or act as a proxy for traffic received from network
    measurement tools.
% paragraph Sender (end)

% Router
% - Receive discovery requests, send discovery responses, forward discovery
%   requests
% - Forward and replicate MEADcast packets
% - MEADcast to unicast transformation
% - L4 Checksum correction
% - Turn MEADcast support on and off
\paragraph{Router} % (fold)
\label{par:Router}
To enable the usage of \gls{mead} the router should encompass the following
    features:

The router should receive discovery requests, reply to them with a discovery
    response, and forward the request according to the destination address.
Additionally, the router needs to receive, process, modify, replicate, and
    forward \gls{mead} data packets accordingly.
For each receiver listed in the \gls{mead} header assigned to the router itself,
    the data should be transmitted the data to the respective recipient using
    IP Unicast.
In case of a \gls{mead} to unicast transformation Layer 4 checksum correction is
    may required.
Lastly, it is preferable, if \gls{mead} can be turned on and off, to ease the
    deployment of various level of network support.
% paragraph Router (end)

\paragraph{Non-functional} % (fold)
\label{par:Non-functional}
Given that the performance evaluation of \gls{mead} encompasses time-sensitive
    metrics such as latency and jitter, it is imperative that the clocks of
    each node are synchronized.
This synchronization is crucial for ensuring comparable measurements.
Additionally, similar routing procedures must be employed for IP Unicast, IP
    multicast, and \gls{mead} to maintain consistency in the evaluation.
For instance, comparing software-based \gls{mead} routing with hardware-based IP
    routing would not yield meaningful results.
% paragraph Non-functional (end)
% subsection Requirements (end)


% subsection Topology (end)
% section T (end)

% \section{Scenarios} % (fold)
% \label{sec:Scenarios}
% % Questions
% % Motivation
% % Characteristics
% % Domains
% % Experiments
% % Parameters
%
% % Char -> Domain
% % To design comparable experiments we have to elaborate characteristics of 
% %     applications that may benefit from Multicast-Communication.
% % Based on these characteristics, we formulate potential application domains for
% %     Multicast-Communication.
% % Lastly, a series of experiments is depicted.
%
% To design comparable experiments we have to elaborate on application domains
%     that may benefit from Multicast-Communication.
% Following, we formulate application characteristics distinguishing these
%     domains.
% This knowledge enables us to design a series of experiments representing
%     a variety of application domains with differing characteristics.
% Thereby we aim to answer RQX.
%
% \subsection{Motivations for employing Multicast} % (fold)
% \label{sub:MotivationsForMulticast}
% To select appropriate measurements and experiments, it is essential to delve
%     into the \textit{motivations} for employing multicast communication and,
%     specifically, MEADcast as a protocol.
%
% % Motivations to use Multicast:
% % Efficient resource usage, to...
% %   - Overcome resource limitation (technical or financial cause)
% %       Technical: no more upstream available,
% %           ip cam can't handle streaming to multiple receivers
% %       Financial: don't want to pay more for: IPS upstream, cloud egress, VPN, ...
% %                   or just reduce costs
% %   - use traffic intense services, despite ones bandwidth limitation
% %   - facilitate new services or improve existing ones
% %   - Emissions
% In a broader context, the decision to utilize multicast often resolves around
%     optimizing resource usage.
% In the absence of resource constraints, one might question opting for multicast
%     rather than ubiquitous unicast communication.
% However, several compelling reasons advocate for the adoption of a multicast
%     protocol.
% A predominant motive is overcoming \textit{resource limitations}, which can
%     manifest technically, such as a network connection incapable of providing
%     the required bandwidth, or \gls{iot} devices like IP cameras struggling
%     to handle numerous unicast connections requesting identical content.
%
% Financial considerations are also a driving reason for employing multicast,
% since most internet services operate on a subscription or pay-as-you-go pricing
%     model (pay per usage).
% \glspl{isp}, mobile phone operators, and VPN providers structure their services
%     based on factors like monthly available traffic volume, upstream and
%     downstream bandwidth.
% Subscribed service levels can thus become the financial cause of resource
%     limitations.
% Cloud providers, in particular, heavily rely on pay-as-you-go pricing, often
%     contending their customers with expensive ``egress costs''.
% Deploying multicast emerges as a strategic approach to not only overcome
%     technical and financial constraints but also to reduce operational costs.
%
% Another motivation for multicast adoption is to facilitate new services or 
%     enhance existing ones.
% For example, employing multicast for delivering multimedia content enables
%     service providers to offer higher-quality audio or video without the need
%     to either increase operational costs or overcome prevailing resource
%     limitations.
% Moreover, multicast can empower individuals to access bandwidth-intensive
%     services despite their limited bandwidth.
% For instance, \gls{p2p} conferencing services could leverage multicast to
%     bridge the asymmetric access link common in most households (high
%     downstream and low upstream bandwidth) \cite{xcast_rfc,cartesian_us_bw}.
% Additionally, network operators may strategically deploy multicast within their
%     domain to decrease overall network load, concurrently reducing operational
%     costs and mitigating emissions caused by \gls{ct} (see
%     \autoref{sec:Motivation}).
%
% % Motivations for choosing MEADcast as Multicast protocol:
% % - IP Multicast often no real option
% % - Limited network control (fallback)
% % - No deployment of specialized client software
% % - Protect receiver privacy
% % - Access control to multicasted content
% As constituted in \autoref{sec:Challenges of Multicast} and
%     \autoref{sub:IP Multicast}, in many scenarios, the usage of IP Multicast
%     is not a viable option.
% This creates a potential application space for MEADcast, particularly in
%     circumstances with limited network control and an uncertain degree of
%     IP Multicast support.
% Another potential domain for MEADcast is in facilitating access control-based
%     multicast communication, addressing the absence of a receiver authorization
%     mechanism in IP Multicast \cite{diot2000deployment}.
% In comparison to alternatives like Xcast or \gls{alm}, MEADcast might be
%     favored in situations where the disclosure of receiver information, such as
%     IP addresses, among other group members, is deemed unacceptable.
% % subsection Motivations (end)


% chapter Design (end)
