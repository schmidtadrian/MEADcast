\chapter{Introduction} % (fold)
\label{chap:Introduction}


\section{Motivation} % (fold)
\label{sec:Motivation}


% what is multicast
% why it makes sense in theory
% still lacks far behind expectations
% several reasons...
Over the last two decades, the number of people with access to the internet has
    continuously increased at a rate of 5\%, resulting in a current total of
    over 5 billion individual internet users \cite{itu_digdev}.
Furthermore, the unrelenting demand for data has led to an average growth in
    total bandwidth usage of 30\% over the past five years.
This development was further accelerated by the COVID-19 pandemic, as the
    increased reliance on remote work, online education, and digital
    entertainment surged the bandwidth consumption \cite{cartesian_us_bw}.
These growth rates are expected to remain high in the future.
Especially Multimedia content like Video Streams, Conferences, and Online Games
    depict a major portion of the global internet traffic.
Video traffic is accountable for more than half of the bandwidth consumed in
    2020 \cite{cartesian_us_bw}.
Network Operators and Service Providers need to comply with the continuously
    rising demand.

Already in the late 1980s, \citeauthor{deering1990multicast} proposed a
    multicast extension to the IP protocol, to facilitate efficient multipoint
    communication (\textit{1:m, m:n})
    \cite{deering1990multicast, rfc1112_ipmc}.
Multicast offers the advantage, to greatly reduce the occupied bandwidth by
    condensing identical traffic into a single stream, targeted towards
    multiple recipients \cite{rfc3376_igmp}.
Routers may replicate packets of that stream at points where the paths towards 
    the receivers diverge.
Many of today's internet services, particularly those with high bandwidth
    demands, may benefit significantly from multicast delivery
    \cite{ratnasamy2006revisiting, meadcast1}.
Besides technical benefits, the adoption of multicast communication could
    reduce the emissions caused by \gls{ct}.
Several studies assert that networks are accountable for a major portion of
    today's \gls{ct} emissions \cite{andrae2015global}.
Furthermore, they forecast strongly increasing energy consumption by networks
    until 2030.
Multicast communication has the potential to lower energy consumption and
    therefore emissions by promoting more efficient use of network resources
    and potentially reducing the need for additional infrastructure.
% section Problem (end)


\section{Challenges of Multicast} % (fold)
\label{sec:Challenges of Multicast}
% Security
%   - Higher comlexity to implement known features like reliability,
%     encryption, authorization, authentication
%   - Increased risk of amplification attacks


% usage on local scope (examples!)
The application of IP-Multicast has yielded mixed results.
Practically all of today's devices support IP-Multicast
    \cite{ratnasamy2006revisiting}.
Furthermore, it is utilized on a local scale, and various protocols like IPv6
    Neighbor Discovery \cite{rfc4861_ipv6_nd}, RIPv2 \cite{rfc2453_rip}, OSPF
    \cite{rfc2328_ospf} and mDNS \cite{rfc6762_mdns} rely on multicast.

However, more than thirty years since the initial proposal of IP-Multicast, its
    global deployment and usage still lags far behind expectations
    \cite{diot2000deployment, ratnasamy2006revisiting}.
Despite the potential advantages of Multicast, the majority of internet traffic
    continues to rely on point-to-point communication, known as
    \textit{unicast}.
There are several reasons for the limited usage of IP-Multicast.

\paragraph{Feasibility} % (fold)
\label{par:Feasibility}
Besides the aforementioned advantages, IP-Multicast entails various technical
    obstacles.
In comparison to unicast, it suffers from an increased technical complexity,
    requiring the interaction of various protocols on multiple network layers
    \cite{ratnasamy2006revisiting,diot2000deployment}.
Furthermore, multipoint communication in general interferes with the
    application of today's widespread security mechanisms like encryption
    \cite{rafaeli2003group_key_mgm}.
Moreover, IPv4-Multicast is based on a fixed class D address space
    \cite{rfc1112_ipmc}, which bears an insufficient number of addresses
    \cite{meadcast2} and a complex routing procedure
    \cite{diot2000deployment,ratnasamy2006revisiting}.
Global IP-Multicast availability is limited because successful packet delivery
    necessitates all routers along the path to support multicast.
However, this scenario is unlikely since many commercial routers come
    preconfigured with IP-Multicast disabled \cite{aruba_doc}.
Moreover, the following paragraph describes, why Network Operators are probably
    not willing to enable it.
% All involved network operators are required to support Intra-Domain,
% paragraph Feasibility (end)

\paragraph{Desirability} % (fold)
\label{par:Desirability}
% General:
%   - by default no built in security & harder to implement todays security
%     standards on top
%   - No explicit effect on end user --> they will not push ISPs
%   - Most traffic is TCP/HTTP which is hard to implement over multicast
%   - Browsers can't handle UDP
%   - Limited use cases

% Network Operators are not interested:
%   - amplification attacks
%   - billing harder
%   - technical complexity not worth effort, unicast is "good enough" (there is a source)

% IP-Multicast, formally known as \gls{igmp}, offers very limited security
%     features.
So far, Network Operators and ISPs have shown limited efforts to deploy
    Multicast within their Administrative Domains
    \cite{diot2000deployment, ratnasamy2006revisiting}.
The increased technical complexity of Multicast requires more extensive
    management efforts.
Additionally, due to the complex routing procedure, infrastructure upgrades may
    be necessary.
Despite its potential for significant bandwidth savings, ISPs seem to assess
    the deployment of IP-Multicast as an unsuitable investment
    \cite{ratnasamy2006revisiting}.
Another reason is the more complex pricing model of multipoint communication.
Charging for multicast services is non-trivial compared to existing unicast
    billing \cite{ratnasamy2006revisiting}.
On top of that, IP-Multicast hampers Network Operators' ability to anticipate
    network load.
Forecasting the number of replicas generated from a multicast packet, entering
    the Network Operators Administrative Domain, is unfeasible
    \cite{diot2000deployment}.
Combined with the limited security mechanisms of IP-Multicast, this represents
    a vulnerability, potentially exploited for amplification attacks.
This fact makes intra-domain IP-Multicast even more unlikely.

Unless ISPs face pressure to expand their service offering, increasing
    multicast deployment is doubtful.
As multicast delivery has no direct impact on receivers, customers are not 
    expected to exert the necessary pressure.
Moreover, statistics illustrate that more than half of the internet traffic
    volume is HTTP-based \cite{cloudflare2023radar}, which is not well suited
    for multipoint communication.
Additionally the usage of Multicast is discouraged by web browsers, the most 
    widely utilized HTTP clients, due to their technical limitation to
    TCP/HTTP\footnotemark.

\footnotetext{
Despite the growing popularity and browser support for QUIC \cite{ietf2021quic,
    fb2020quic}, also known as HTTP over UDP, it encounters similar challenges
    as TCP/HTTPS.
This include issues such as packet acknowledgment and the client side
    generation of random numbers \cite{rfc9000_quic}.
}
% paragraph Desirability (end)
% section Challenges of Multicast (end)


\section{Goal and Contribution} % (fold)
\label{sec:Contribution}
% Current limitations --> alternatives lending to multicast (ALM, Xcast, MEADcast)
% Investigate MEADcast
% As suggested in the paper kernel implementation follows simulation & SDN
% Compare the implementation with uni- & multicast
% verify if MEADcast really works over internet (feasibility)
% Provide some suggestion, for which scenarios the usage of MEADcast is sensible

% Router Implementation
% Feasibility study of using MEADcast over the internet due to concerns
%   regarding extension header processing
% Performance comparison to uni- and multicast
% Identify scenarios & application categories the usage of MEADcast is sensible

% Evaluation Goals
% Studying the feasibility of deploying MEADcast in a real-network & the internet
%   (limitations or structural problems of the protocol design)
% Conducting a performance comparison to IP uni- and multicast
% Identify scenarios & application categories/characteristics justifying the usage of MEADcast

The current state of the internet put forth various unicast-based alternatives,
    aimed at addressing the absence of a globally usable multicast protocol.
One such alternative, known as MEADcast \cite{meadcast1}, offers the capability
    for 1:n sender-based IPv6 multipoint communication over the internet
    \cite{meadcast2}.
Key features of MEADcast include the preservation of receiver privacy,
    technology-agnostic destinations, and zero network support requirements.

Building upon prior research conducted by \citeauthor{meadcast2}
    \cite{meadcast2}, the primary goal of this thesis is to conduct a
    real-world evaluation of MEADcast, utilizing a Linux Kernel implementation
    of the required router software.
This step represents a logical progression from earlier investigations of
    MEADcast, which were based on network simulations \cite{meadcast1} and
    \gls{sdn} \cite{sdn_ba}.
The evaluation primarily focuses on the following aspects:

\paragraph{Feasibility Study} % (fold)
\label{par:Feasibility Study}
The first part of the evaluation assesses the feasibility of deploying
    MEADcast in a real network.
This aims to identify potential limitations and structural issues of the
    current protocol specification.
Further investigation examines the practicality of using MEADcast on the
    internet, taking into consideration concerns related to IPv6 extension
    header processing \cite{rfc7872_ext_hdrs_drop_rate}.
% paragraph Feasibility Study (end)

\paragraph{Performance Evaluation} % (fold)
\label{par:Performance Evaluation}
A comparative performance analysis is conducted to evaluate MEADcast in
    comparison to both IP unicast and multicast.
This assessment provides insights into the efficiency and effectiveness of
    MEADcast as a multipoint communication solution.
% paragraph Performance Evaluation (end)

\paragraph{Scenario identification} % (fold)
\label{par:Scenario identification}
Building on the results of the previous evaluation steps, this phase involves
    identifying scenarios, application categories, and characteristics that
    justify the utilization of MEADcast.
Thereby, it can be determined where MEADcast may offer advantages and excel in
    real-world applications.
% paragraph Scenario identification (end)
    
% TODO: The formatting of this doesn't look nice. However, so far I have no
% better idea
\noindent\\
Based on the defined objectives, this thesis makes the following contributions:
\begin{enumerate}[itemsep=0.3em]
    \item Development of a Linux Kernel implementation of the MEADcast router.
    \item Implementation of a traffic generator serving as the MEADcast sender.
    \item Deployment of MEADcast in both a testbed and a real network.
    \item Evaluation of MEADcast's feasibility, performance, and potential
            application scenarios, derived from the conducted experiments.
\end{enumerate}

The findings indicate, that with certain limitations, MEADcast is applicable in
    real networks.
The current protocol specification suffers from poor availability on the
    internet due to the limited processing of IPv6 packets with extension
    headers.
However, the proposed modification effectively addresses this obstacle while
    enhancing receiver privacy.
Our measurements suggest that MEADcast's performance falls between uni- and
    multicast, particularly showing promise in scenarios characterized by
    limited bandwidth and network control.

% section Contribution (end)


\section{Method} % (fold)
\label{sec:Method}
To ensure the achievement of the previously defined goals this thesis follows
    the procedure illustrated in \autoref{fig:method}.
% First, we conduct a literature review to analyze the current challenges of
%     IP-Multicast.
First, we conduct a literature review of several multicast protocols and
    analyze the current challenges of global multipoint communication.
With a clear understanding of the relevant protocols and their limitations we
    define the objectives of this thesis.
% Based on these results, we define the objectives of this thesis.
Furthermore, we select adequate evaluation metrics and criteria to asses
    MEADcast's feasibility, performance, and potential application scenarios.
Subsequently, we design experiments aimed at capturing these metrics.
Next, we outline the testbed requirements derived from the experiments and
    elaborate on a corresponding testbed design.
% Next, we derive requirements from the experiments and elaborate on a testbed
%     design.
MEADcast is deployed in the testbed and the experiments are conducted.
Finally, we present the obtained results.
These findings are critically evaluated with respect to the thesis' overarching
    goal, providing valuable insights into the feasibility and performance of
    MEADcast.
This analysis allows us to identify scenarios and applications for which the
    protocol is well-suited.
% This analysis opens the door to identifying scenarios and applications the
%     protocol is well suited for.


\begin{figure}
\centering
\begin{tikzpicture}
[auto,font=\footnotesize,
% STYLES
every node/.style={node distance=1.2cm},
force/.style={rectangle, rounded corners, draw, fill=black!8, text width=2.75cm, text badly centered, minimum height=1cm, font=\footnotesize\sffamily}]

\node [force]                                                           (problem)       {Problem analysis};
\node [force, below of=problem,         right=-1.35cm of problem]       (goal)          {Goal definition};
\node [force, below of=goal,            right=-1.35cm of goal]          (criteria)      {Metric \& Criteria selection};
\node [force, below of=criteria,        right=-1.35cm of criteria]      (design)        {Experiment design};
\node [force, below of=design,          right=-1.35cm of design]        (requirements)  {Requirements analysis};
\node [force, below of=requirements,    right=-1.35cm of requirements]  (testbed)       {Testbed design};
\node [force, below of=testbed,         right=-1.35cm of testbed]       (experiment)    {Experiment conduction};
\node [force, below of=experiment,      right=-1.35cm of experiment]    (evaluation)    {Evaluation};

\draw[-latex,thick] (problem.south west)        +(7mm,0) |- (goal.west);
\draw[-latex,thick] (goal.south west)           +(7mm,0) |- (criteria.west);
\draw[-latex,thick] (criteria.south west)       +(7mm,0) |- (design.west);
\draw[-latex,thick] (design.south west)         +(7mm,0) |- (requirements.west);
\draw[-latex,thick] (requirements.south west)   +(7mm,0) |- (testbed.west);
\draw[-latex,thick] (testbed.south west)        +(7mm,0) |- (experiment.west);
\draw[-latex,thick] (experiment.south west)     +(7mm,0) |- (evaluation.west);
\end{tikzpicture}

\caption{Research method} \label{fig:method}
\end{figure}

% section Method (end)

\section{Structure} % (fold)
\label{sec:Structure}
The method employed in this thesis ensures a cohesive narrative, with each
    chapter building upon the knowledge acquired in the previous sections.
\autoref{chap:Introduction} depicts the motivation for investigating multipoint
    communication, highlights the current challenges of IP-Multicast, and
    articulates the goal of this thesis.
Moving forward, \autoref{chap:Background and Work} establishes the theoretical
    foundation for subsequent investigations, by examining several multicast
    protocols, including IP-Multicast, Xcast, and MEADcast.
Additionally, the chapter provides a brief overview of Kernel development
    fundamentals and the network stack, laying the groundwork for the router
    implementation.
\autoref{chap:Design} presents our selection of evaluation metrics and
    criteria, outlines the testbed requirements and presents the corresponding
    testbed design.
\autoref{chap:Implementation} delves into technical details about the Kernel
    implementation of the router software and provides detailed specifications
    of the testbed.
This chapter serves as the foundation for the practical experiments.
In \autoref{chap:Evaluation} we present the results from our experiments and
    evaluate them.
Finally, \autoref{chap:Summary} summarizes the findings and draws conclusions
    from this research.
Additionally, it outlines potential avenues for future work and exploration in
    this field.
% section Structure (end)

% chapter Introduction (end)
