\chapter{Background and Related Work} % (fold)
\label{chap:Background and Work}

\section{Multicast Protocols} % (fold)
\label{sec:Mutlicast Protocols}
Multicast is a technique of sending data one-to-many or many-to-many.
Related technologies are unicast, broadcast and anycast.
% section Mutlicast (end)

\subsection{IP Multicast} % (fold)
\label{sub:IP Multicast}
\begin{itemize}\itemsep0em
    \item Separate address space
    \item IGMP / MLD
    \item Intra-Domain Routing: DVMRP, MOSPF, PIM dense/spare
    \item Inter-Domain Routing: MBGP, MSDP
\end{itemize}
% subsection IP Multicast (end)


% \subsection{Multicast over Unicast} % (fold)
% \label{sub:Multicast over Unicast}
% \begin{itemize}\itemsep0em
%     \item Xcast family (Xcast, Xcast+, GXcast, Xcast6 Treemap (island))
%     \item MEADcast
%     \item Bier
% \end{itemize}

\subsection{Xcast}
\label{sub:Xcast}
Traditional multicast scales well with large multicast groups but have issues
with a high number of distinct groups.
Xcast is a mutlicast protocol with complementary scaling properties compared to
the traditional approach \cite{xcast_rfc}.
The protocol is designed with the key idea of supporting huge numbers of small
multicast sessions.
Xcast achieves this by explicitly encoding the receiver addresses in each
packet, instead of using a multicast addresses \cite{xcast_rfc}.

%%%%%%%%% XCast %%%%%%%%%
% many small groups
% no multicast ips --> no multicast routing
% no per-session signaling and state
% routers must be Xcast capabale or tunneling
% along each path must be an Xcast capabale node (may receiver)

% hdr
% ip
% src: sender
% dst: all_xcast_addr (multicast addrs routers must join)
% xcast
% dsts list of addrs
% protocol id of next header
% bitmap that indicates which addresses needs to be processed
% optional list of ports

%%%%%%%%% XCast+ %%%%%%%%%
% incorporates host group model and Xcast
% enhances routing efficiency
% each sender/receiver has a designated router
% to join a multicast group a receiver sends an IGMP. The DRs intercepts the
% message and sends an registration request msg to the sender. At the senders
% side this msg is intercepted by its DR
% S <--MC--> DR <--XC--> DR <--MC--> R
% Only diff to Xcast is, that the Receivers DRs addresses are encoded in Xcast
% dsts field
% --> better than Xcast, if multiple receivers per DR
% --> DRs keep track, which clients joined the Xcast group (usally via MC)

%%%%%%%%% GXCast %%%%%%%%%
% subsection Xcast (end)

\subsection{MEADcast} % (fold)
\label{sub:MEADcast}

% subsection MEADcast (end)

\section{Linux Kernel} % (fold)
\label{sec:Linux Kernel}

\subsection{Fundamentals} % (fold)
\label{sub:Fundamentals}

% subsection Fundamentals (end)

\subsection{Network Stack} % (fold)
\label{sub:Network Stack}

% subsection Network Stack (end)
% section Linux Kernel Network Internals (end)

% chapter Background Work (end)
