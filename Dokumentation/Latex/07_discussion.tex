\chapter{Discussion} % (fold)
\label{chap:Discussion}

% Research Questions:

% Scenario identification
% - When to use MEADcast (Application Requirements, Network Environment)
% - No disadvantage in using MEADcast (bc of Fallback)
% - Limited network control, deploying a single router at key location can already signifiantly
%     improve performance (ref to r03)
% - 


% How does MEADcast perform compared to IP-Unicast and IP-Multicast?
% - throughput, latency, jitter, and resource utilization
% - Impact of discovery phase (Overhead, Shift from extensive unicast to MEADcast delivery)
% - Couldn't measure resource utilization but we argue less processing bc of
%   lower latency and drop rate for large number of receivers

% Which application and characteristics are well served by MEADcast?

% In which conditions is the usage of MEADcast sensible?
% - Prevailing circumstances, level of network control, endpoint distribution



% Performance very good, can reduce netload, bandwidth, time, latency
% Negligible effect of the discovery phase on bandwidth and latency.
%   We recommend relative low interval
% Splitting of packets after the first hop can be an issue
% Even in cases of higher endpoint distribution and one client per router
%   performance improvement. As long as a few links are shared
% Grouping gives the sender the ability to adjust the protocol to his needs.
    % large groups --> general higher bw savings, more clients effected by pkt loss
    % discovery interval --> often: latency spikes more often, faster recovery
    % merging nodes --> better performance for higher distribution of nodes
% Because of the resilience and adaptivity to changing 

This chapter evaluates the results from our experiments presented in
    \autoref{chap:Evaluation} and aims to address the research questions
    formulated in \autoref{sec:Measurements}.
As outlined in \autoref{sec:Contribution}, the primary goal of this thesis is
    to conduct a real-world evaluation of MEADcast, focusing on the aspects of
    \textit{feasibility}, \textit{performance}, and \textit{scenario
    identification}.

\paragraph{Feasibility} % (fold)
\label{par:discussion_Feasibility}
\begin{itemize}
\item[\textit{RQ1}]
    \textit{How robust is the current MEADcast specification? (deployment
        limitations \& structural issues of the protocol specification)}\par
% - Feasibility of deploying MEADcast (limitations & structural issues of specification)
%%%
% - Correct routing header
% - Omit Hop-by-Hop header
% - Each router interface needs an IP address
% - Packet gets split, which can lead to an increase in bandwidth utilization
% - If MEADcast router fails, IP addresses can be leaked
% - PLUS that no special routing is required (was hard to make Multicast routing)
    \textbf{Protocol specification:}
    As discussed in \autoref{sec:Protocol Specification}, we propose the
        omission of the empty Hop-by-Hop IPv6 Extension, which experiences an
        increased drop rate \cite{rfc7872_ext_hdrs_drop_rate}, aiming to reduce
        the protocol's overhead by removing a header that serves no purpose,
        decrease the likelihood of slow path processing, and increase the
        probability of being forwarded by non-MEADcast routers.
    Furthermore, to align with RFC 8200 \cite{rfc8200_ipv6_hdr} and mitigate
        the risk of intermediate nodes dropping MEADcast packets due to a
        malformed IPv6 routing extension, we advocate the inclusion of the
        ``Segments Left'' field from the static IPv6 routing header extension.

    \textbf{Deployment:}
    The conducted series of experiments, encompassing deployment and evaluation,
        effectively showcases the feasibility of employing MEADcast within a
        medium-sized network.
    MEADcast deployment requires only the installation of sender and router
        software.
    Additionally, the fallback mechanism enables MEADcast to operate even
        without dedicated router support, presenting a distinct advantage over
        IP Multicast.
    In contrast to IP Multicast, which entails increased technical complexity
        and a compound routing procedure, necessitating all routers to support
        the protocol (see \autoref{sub:IP Multicast}), MEADcast imposes no
        additional requirements beyond the sender and router software.
    This characteristic facilitates a partial deployment of MEADcast,
        underscoring its superior feasibility compared to IP Multicast.

% How robust is the current specification?
% - deliberate routing changes
% - network disruption (link & router outage)
% - Anomaly handling
% - firewall (fallback mechanism)
% => adaptivity and suitability for dynamic network environments)
%%%%
% - Handles routing changes, and network disruption similar to Unicast.
%   Needs max. one discovery phase afterwards
% - Anomaly handling mostly implementation dependent (no router authorization so far)
% - Falling back to MEADcast if possible lead to way better results
    \textbf{Robustness:}
    MEADcast has demonstrated resilience to deliberate routing changes, network
        disruptions such as link and router outages, and packet discarding by
        intermediate nodes.
    As MEADcast operates based on IP Unicast routes, its adaptability to
        evolving network topologies primarily relies on the underlying routing
        protocol.
    However, if a packet's route is altered and it does not traverse the
        routers listed in its header, delivery disruption persists until the
        next discovery phase.
    In the event of a firewall dropping packets during MEADcast transmission,
        disruptions endure until the sender detects the modified topology tree
        in the subsequent discovery phase.
    Both MEADcast and IP Unicast fallback effectively address the presence of
        a firewall.
    However, reverting to a router positioned in front of the firewall results
        in 50\% less network bandwidth utilization compared to IP Unicast.

    \textbf{Anomaly Handling:}
    In cases where a router fails to perform the MEADcast to IP Unicast
        transformation, packets are forwarded to the client specified in the IP
        destination field, potentially exposing group member IP addresses.
    The handling of anomalies is implementation-specific.
    Since no authentication mechanism is specified for MEADcast, miscellaneous
        discovery responses can be injected, potentially hindering transmission
        to multiple clients.

    \textbf{Packet replication:}
    The experiments have revealed an inefficiency within the MEADcast routing
        process.
    Specially, when multiple router addresses are included within a single
        packet, sharing a common path of MEADcast routers, the first MEADcast
        hop generates a replica for each router in the address list.
    This behavior, while technically correct, arises from the stateless nature
        of routers, which lack information regarding whether routers from the
        address list share another intermediate router that could instead
        perform the replica generation.
    Although the sender possesses knowledge of how long routers within a packet
        share the same path, the current MEADcast specification lacks a feature
        to determine the point of packet replication, thus preventing previous
        MEADcast routers from doing so.
    As depicted in \autoref{fig:link_bw_l2l3_100} this inefficiency leads to a
        significant increase in bandwidth utilization.
    To address this issue, we propose the introduction of a ``Don't Replicate''
        field in the header.
    Similar to the IPv6 Hop Limit, this field contains a counter that
        decrements with each forwarding MEADcast router.
    As long as the field is greater than 0, the packet should not be
        replicated, facilitating the sender to mitigate this inefficiency.

    These results emphasize the feasibility of employing MEADcast in
        medium-sized networks.
    Moreover, the experiments illustrate MEADcast's resilience to network
        disruptions and recovery capabilities.
    This highlights the protocol's adaptivity and suitability for dynamic
        network environments, offering promising initial insights into the
        real-world applicability and resilience of MEADcast.
    However, anomaly handling is highly implementation specific.
    % propose several adaption to the MEADcast header, to comply with existing
    % RFCs, reduce overhead, and mitigate routing inefficiencies.
\end{itemize}
% paragraph Feasibility (end)

\paragraph{Performance} % (fold)
\label{par:discussion_Performance}
\begin{itemize}
\item[\textit{RQ2}]
    \textit{How does MEADcast perform compared to IP-Unicast and IP-Multicast?}
% Performance Evaluation:
% - Compare MEADcast with uni and multicast (efficiency and effectiveness)
%%%%%
% Performance falls always between Unicast and Multicast
% AVG Netload Reduction 56%
% AVG Upstream Reduction 81.23%
% AVG Transfer Time Reduction (EX2) 49.18%
    
\end{itemize}

% paragraph Performance (end)

\paragraph{Scenario identification} % (fold)
\label{par:discussion_scenario}
\begin{itemize}
\item[\textit{RQ3}]
    \textit{Which applications and characteristics are well served by MEADcast?}
\item[\textit{RQ4}]
    \textit{In which conditions is the usage of MEADcast sensible?}
    % limited resources
    % limited network control
\end{itemize}
% paragraph  (end)


% chapter Discussion (end)
