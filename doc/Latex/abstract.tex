\vspace*{2cm}

\begin{center}
    \textbf{Abstract}
\end{center}

\vspace*{1cm}
\noindent
Despite ever-increasing bandwidth demands and the compatibility of numerous
    internet services with multicast communication, the deployment of IP
    Multicast has not met expectations.
Consequently, this thesis evaluates a novel multicast protocol named MEADcast,
    focusing on its feasibility, performance, and potential application domains.
% Method
The evaluation utilizes a Linux Kernel implementation of the router along
    with a standalone implementation of the sender.
Four distinct use cases, each representing real-world scenarios potentially
    benefiting from multicast communication, are examined.
% Results / Discussion
% Feasibility
The investigation demonstrates the feasibility of deploying MEADcast in
    medium-sized networks, offering promising initial insights into the
    real-world applicability of MEADcast.
MEADcast has proven its adaptivity and suitability for dynamic network
    environments, showcasing resilience and recovery capabilities in response
    to network disruptions, particularly attributable to its well-functioning
    fallback mechanism.
% Performance
MEADcast deployment results in significant performance improvements, including
    a 56\% reduction in network bandwidth utilization, an 81.23\% decrease in
    sender upstream bandwidth consumption, and a 49.18\% reduction in total
    transfer time.
Overhead generated by the discovery phase primarily originates from the initial
    discovery, with subsequent recurring discovery phases incurring negligible
    overhead.
% Scenarios / Conditions
MEADcast is well-suited for both applications characterized by communication
    patterns of steady flows (e.g. multimedia stream) and recurring bursts
    (e.g. file transfer).
The protocol excels in scenarios characterized by resource constraints and
    limited network control.
Deploying a single MEADcast router at a strategic location can significantly
    reduce bandwidth utilization between the sender and router.
However, increasing the number of MEADcast routers may not necessarily enhance
    performance due to packet replication inefficiencies.
% Recommendation
Given its resilient fallback mechanism and support for partial deployment,
    MEADcast adoption is advocated whenever multicast communication is
    applicable but IP Multicast is unavailable.
% Revision
Nonetheless, several refinements are proposed for the upcoming MEADcast
    revision, aiming to address existing inefficiencies and mitigate malicious
    interference.
