\chapter{Discussion} % (fold)
\label{chap:Discussion}

% Research Questions:

% Scenario identification
% - When to use MEADcast (Application Requirements, Network Environment)
% - No disadvantage in using MEADcast (bc of Fallback)
% - Limited network control, deploying a single router at key location can already signifiantly
%     improve performance (ref to r03)
% - 


% How does MEADcast perform compared to IP Unicast and IP Multicast?
% - throughput, latency, jitter, and resource utilization
% - Impact of discovery phase (Overhead, Shift from extensive unicast to MEADcast delivery)
% - Couldn't measure resource utilization but we argue less processing bc of
%   lower latency and drop rate for large number of receivers

% Which application and characteristics are well served by MEADcast?

% In which conditions is the usage of MEADcast sensible?
% - Prevailing circumstances, level of network control, endpoint distribution



% Performance very good, can reduce netload, bandwidth, time, latency
% Negligible effect of the discovery phase on bandwidth and latency.
%   We recommend relative low interval
% Splitting of packets after the first hop can be an issue
% Even in cases of higher endpoint distribution and one client per router
%   performance improvement. As long as a few links are shared
% Grouping gives the sender the ability to adjust the protocol to his needs.
    % large groups --> general higher bw savings, more clients effected by pkt loss
    % discovery interval --> often: latency spikes more often, faster recovery
    % merging nodes --> better performance for higher distribution of nodes
% Because of the resilience and adaptivity to changing 

This chapter evaluates the results from our experiments presented in
    \autoref{chap:Evaluation} and aims to address the research questions
    formulated in \autoref{sec:Measurements}.
As outlined in \autoref{sec:Contribution}, the primary goal of this thesis is
    to conduct a real-world evaluation of MEADcast, focusing on the aspects of
    \textit{feasibility}, \textit{performance}, and \textit{scenario
    identification}.

\paragraph{Feasibility} % (fold)
\label{par:discussion_Feasibility}
\begin{itemize}
\item[\textit{RQ1}]
    \textit{How robust is the current MEADcast specification? (deployment
        limitations \& structural issues of the protocol specification)}\par
% - Feasibility of deploying MEADcast (limitations & structural issues of specification)
%%%
% - Correct routing header
% - Omit Hop-by-Hop header
% - Each router interface needs an IP address
% - Packet gets split, which can lead to an increase in bandwidth utilization
% - If MEADcast router fails, IP addresses can be leaked
% - PLUS that no special routing is required (was hard to make Multicast routing)
    \textbf{Protocol specification:}
    As discussed in \autoref{sec:Protocol Specification}, we propose the
        omission of the empty Hop-by-Hop IPv6 Extension, which experiences an
        increased drop rate \cite{rfc7872_ext_hdrs_drop_rate}.
    This adjustment aims to reduce the protocol's overhead by eliminating a
        header that serves no purpose, decrease the likelihood of slow path
        processing, and increase the probability of being forwarded by
        non-MEADcast routers.
    Furthermore, to align with RFC 8200 \cite{rfc8200_ipv6_hdr} and mitigate
        the risk of intermediate nodes dropping MEADcast packets due to a
        malformed IPv6 routing extension, we advocate for the inclusion of the
        ``Segments Left'' field from the static IPv6 routing header extension.

    \textbf{Deployment:}
    The conducted series of experiments, encompassing deployment and evaluation,
        effectively demonstrates the feasibility of employing MEADcast within a
        medium-sized network.
    MEADcast deployment requires only the installation of sender and router
        software.
    Additionally, the fallback mechanism enables MEADcast to operate even
        without dedicated router support, presenting a distinct advantage over
        IP Multicast.
    In contrast to IP Multicast, which entails increased technical complexity
        and a compound routing procedure, necessitating all routers to support
        the protocol (see \autoref{sub:IP Multicast}), MEADcast imposes no
        additional requirements beyond the sender and router software.
    This characteristic facilitates a partial deployment of MEADcast,
        underscoring its superior feasibility compared to IP Multicast.

% How robust is the current specification?
% - deliberate routing changes
% - network disruption (link & router outage)
% - Anomaly handling
% - firewall (fallback mechanism)
% => adaptivity and suitability for dynamic network environments)
%%%%
% - Handles routing changes, and network disruption similar to Unicast.
%   Needs max. one discovery phase afterwards
% - Anomaly handling mostly implementation dependent (no router authorization so far)
% - Falling back to MEADcast if possible lead to way better results
    \textbf{Robustness:}
    MEADcast has demonstrated resilience and recovery from deliberate routing
        changes, network disruptions such as link and router outages, and
        packet discarding by intermediate nodes.
    As MEADcast operates based on IP Unicast routes, its adaptability to
        evolving network topologies primarily relies on the underlying routing
        protocol.
    However, if a packet's route is altered and it does not traverse the
        routers listed in its header, delivery disruption persists until the
        next discovery phase.
    In the event of a firewall dropping packets during MEADcast transmission,
        disruptions endure until the sender detects the modified topology tree
        in the subsequent discovery phase.
    Both MEADcast and IP Unicast fallback effectively address the presence of
        a firewall.
    However, reverting to a router positioned in front of the firewall results
        in 50\% less network bandwidth utilization compared to IP Unicast.

    \textbf{Anomaly Handling:}
    In cases where a router fails to perform the MEADcast to IP Unicast
        transformation, packets are forwarded to the client specified in the IP
        destination field, potentially exposing group member IP addresses.
    We recommend investigating whether targeting packets to the next MEADcast
        hop, rather than the first client from the address list, can prevent
        the exposure of group member information in cases of failure.
    This investigation should also consider the implications in cases of route
        changes and network disruptions.
    It is important to note that the handling of anomalous discovery responses
        is implementation-specific.
    Since no authentication mechanism is specified for MEADcast, malicious
        discovery responses can be injected, potentially hindering transmission
        to multiple clients.

    \textbf{Packet replication:}
    The experiments have revealed an inefficiency within the MEADcast routing
        process.
    Specially, when multiple router addresses are included within a single
        packet, sharing a common path of MEADcast routers, the first MEADcast
        hop generates a replica for each router in the address list.
    This behavior, while technically correct, arises from the stateless nature
        of routers, which lack information regarding whether routers from the
        address list share another intermediate router that could instead
        perform the replica generation.
    Although the sender possesses knowledge of how long routers within a packet
        share the same path, the current MEADcast specification lacks a feature
        to determine the point of packet replication, thus preventing previous
        MEADcast routers from doing so.
    As depicted in \autoref{fig:link_bw_l2l3_100} this inefficiency leads to a
        significant increase in bandwidth utilization.
    To address this issue, we propose the introduction of a ``Don't Replicate''
        field in the header.
    Similar to the IPv6 Hop Limit, this field contains a counter that
        decrements with each forwarding MEADcast router.
    As long as the field is greater than 0, the packet should not be
        replicated, facilitating the sender to mitigate this inefficiency.

    These results emphasize the feasibility of employing MEADcast in
        medium-sized networks.
    Moreover, the experiments illustrate MEADcast's resilience to network
        disruptions and recovery capabilities.
    This highlights the protocol's adaptivity and suitability for dynamic
        network environments, offering promising initial insights into the
        real-world applicability and resilience of MEADcast.
    However, anomaly handling is highly implementation-specific.
    Furthermore, we propose several refinements to the MEADcast specification
        and further investigation of their implications.
\end{itemize}
% paragraph Feasibility (end)

\paragraph{Performance} % (fold)
\label{par:discussion_Performance}
\begin{itemize}
\item[\textit{RQ2}]
    \textit{How does MEADcast perform compared to IP Unicast and IP Multicast?}
% Performance Evaluation:
% - Compare MEADcast with uni and multicast (efficiency and effectiveness)
%%%%%
    
% Resource utilization
%   - Sender load (See increase in transfer time EX1 4-5 clients for unicast)
%   - Latency and Drop rates grow 

    % NET BW, UP BW, TIME
    % Performance falls always between Unicast and Multicast
    % AVG Netload Reduction: Mead 56%, Multi 82.87% --> 32.42%
    % AVG Upstream Reduction: Mead 81.23%, Multi 96.15% --> 15.51%
    % AVG Transfer Time Reduction (EX2): Mead 49.18%, Multi 67.34% --> 26.97%
    The experiments emphasize that MEADcast significantly enhances performance
        metrics such as total network bandwidth utilization, sender upstream
        bandwidth utilization, and total transfer time compared to IP Unicast.
    On average, MEADcast reduced network bandwidth utilization by more than
        half (56\%), sender upstream bandwidth utilization by 81.23\%, and
        total transfer time by half (EX2: 49.18\%).
    In contrast, compared to IP Multicast, MEADcast increased network bandwidth
        utilization by 32.42\%, sender upstream bandwidth by 15.51\%, and total
        transfer time by 26.97\% (EX2).

    % Implications of the discovery phase
    \autoref{sub:Results_Effects of the Discovery phase} illustrates that the discovery
        phase produces an average overhead of 4.69\% in total network bandwidth
        utilization, 18.36\% in sender upstream bandwidth utilization, 0.88\%
        (EX2) in total transfer time, and 39.57\% in average latency.
    However, excluding the initial discovery phase signififantly reduces the 
        average overhead to an increase of 0.62\% in total network bandwidth
        utilization, 0.42\% in sender upstream bandwidth utilization, and
        15.69\% in average latency.

    % Resource utilization
    % We argue, that MEADcast results in a decrease in resouce utilization:
    % - for 4-5 clients Unicast latency strongly increases and packet loss starts
    %   to occurr.
    % - In contrast MEADcast latency is way lower and no or small packet loss
    % - Further we argue that the increase in latency for MEADcast with discovery
    %   supports that.
    % - bc for small numbers the diff in latency between with and without init
    %   discovery is ~11% and for high number, the overhead in latency is 75% with init discovery,
    %   showing Unicast processing utilizies the available resources.
    Although meaningful results for sender and router resource utilization were
        not attainable, we contend that MEADcast results in reduced resource
        utilization compared to IP Unicast as the number of receivers
        increases.
    \autoref{fig:latency_cmp} illustrates a distinct increase in average
        latency for IP Unicast transmissions with more than three clients,
        coinciding with the onset of packet loss.
    This suggests, that IP Unicast transmission exhausts the testbed's
        resources at this point.
    In contrast, MEADcast maintains constant latency levels with little to no
        packet loss.
    Moreover, the observed increase in average latency for MEADcast with
        discovery phase supports our hypothesis.
    As indicated in \autoref{tab:init_dcvr_latency}, the difference in latency
        increase with and without consideration of the initial discovery phase
        is only 11\% for three or fewer clients but rises to 75\% for more than
        three clients.
    This suggests that IP Unicast transmission during the initial discovery
        phase is the primary cause for resource strain.

    % Shift from extensive IP Unicast to MEADcast is possible,
    % leads to significant performance improvements, especially bw and latency reduction
    % results also in less resource utilization
    % Initial discovery causes major portion of the overhead produced by the
    % discovery mechanism, overhead of the recurrying disovery phase is negligible
    The results emphasize the feasibility of a graduate shift from extensive
        IP Unicast to MEADcast delivery, highlighting significant performance
        improvements, particularly in terms of reduced total network and sender
        upstream bandwidth utilization, as well as latency.
    Furthermore, the overhead generated by the discovery phase is primarily
        originates from the initial discovery, whereas the overhead of
        subsequent recurring discovery phases is negligible.
% Effect of the discovery phase
\end{itemize}

% paragraph Performance (end)

\paragraph{Scenario identification} % (fold)
\label{par:discussion_scenario}
\begin{itemize}
\item[\textit{RQ3}]
    \textit{Which applications and characteristics are well served by MEADcast?}
    % File transfer
    % Stream
    % Discovery Phase can impede latency
    % Group:
    % - all tested group sizes from small (<10) up to <= 70 always beneficial
    % - the more members the better
    % - the longer the better, bc negligible overhead of subsequent discovery
    % Communication:
    % - recurring burst (e.g. File transfer) if long enough 
    % - steady stream (e.g. video/audio stream)
    % - not able to test asymmetric communication, but results from symmetric communication should be applicable
    % Network:
    % - No effect on Jitter measurable
    % - MEADcast could reduce latency, however may peak during discovery
    % - throughput intense is well-suited.

    Across all experiments, MEADcast has consistently demonstrated its
        benefits.

    % Group (Members, Distribution, Session length)
    From small group sizes ($<10$) to groups with $\leq 70$ receivers, MEADcast
        has shown improvements in all metrics.
    As the total number of group members and the number of receivers per
        router increased, these enhancements became more pronounced.
    Additionally, MEADcast exhibits performance improvements across all tested
        session durations.
    However, due to the overhead associated with the initial discovery phase,
        we recommend session durations of at least 5 seconds.
    Particularly with longer session durations, the performance improvements
        become more significant, as subsequent discovery phases represent a
        negligible overhead.
    % It is expected that there is an upper limit

    % Communication pattern (burst, steady, symmetric, asymmetric)
    MEADcast is well-suited for communication patterns characterized by steady
        flows (e.g. video/audio stream) and recurring bursts (e.g. file
        transfer).
    Although experiments with asymmetric communication patterns were not
        feasible due to resource limitations, we anticipate that the results
        from symmetric communication patterns are applicable, except for the
        enlarged traffic volume inherent in \gls{p2p} communication.

    % Network
    MEADcast proves beneficial across all group sizes, receiver distributions,
        session durations, and communication patterns.
    However, as group sizes, receiver clustering, and session duration
        increase, the performance improvements compared to IP Unicast become
        more pronounced.
    MEADcast excels in throughput-intense scenarios, with no discernible impact
        on jitter, making it suitable for applications sensitive to jitter.
    However, MEADcast may not be suitable for applications sensitive to initial
        startup latency.

\item[\textit{RQ4}]
    \textit{In which conditions is the usage of MEADcast sensible?}
    % limited resources
    % - especially upstream
    % limited network control
    % - deployment of router at key location can significantly reduce bw
    % offers sender the ability to shape how its packets should be transferred
    % - more MEADcast router does not mean better performance
    % - No disadvantage in deploying MEADcast, offers significant performance
    %   improvement already with a small number of receivers, and routers. And
    %   otherwise just falls back to Uni Cast
    % - If bandwidth should be reduced, but can not ensure Multicast support

    The employment of MEADcast proves particularly beneficial when the
        reduction of the total network bandwidth or sender upstream utilization
        is a major concern, especially in cases where IP Multicast support
        cannot be guaranteed.
    Deploying a single MEADcast router at a strategic location can
        significantly reduce bandwidth utilization between the sender and
        router.
    However, it is important to note that the presence of a higher number of
        MEADcast routers may not necessarily translate to performance
        enhancements, as indicated in \autoref{par:discussion_Feasibility}
        highlighting packet replication inefficiencies.

    MEADcast excels in scenarios characterized by resource constraints and
        limited network control.
    Moreover, MEADcast empowers the sender to granularly tailor the traffic
        pattern according to specific requirements and prevailing network
        conditions.
    Given the protocol's support for partial deployment and resilient fallback
        mechanism, coupled with the absence of any major performance
        disadvantages, we advocate for the adoption of MEADcast whenever
        multicast communication is applicable.
    However, in scenarios where IP Multicast usage can be assured, it remains
        the preferred choice.
    % only disadvantage if MULTIcast works
    % to small can be prevented

\end{itemize}
% paragraph  (end)


% chapter Discussion (end)
