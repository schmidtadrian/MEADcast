\chapter{Code} % (fold)
\label{chap:Code}


\begin{listing}[h]
\begin{minted}[linenos, frame=lines, framesep=2mm, breaklines, fontsize=\small]{bash}
sudo python3 mk_topo.py \
    -t "template/testbed.json" \
    -o "out/" \
    -n "template/net.xml" \
    -u user \
    --ssh "~/id_rsa_vms.pub" \
    --client-run "template/run.sh template/run_client.sh" \
    --router-run "template/run.sh template/run_router.sh" \
    --backing-file "/var/lib/libvirt/images/templates/debian-12-nocloud-amd64.raw" \
    -c "template/chrony.conf" \
    -f "template/frr.conf" \
    -k kernel=/var/lib/libvirt/images/kernel/vmlinuz-6.5.2-dirty, initrd=/var/lib/libvirt/images/kernel/initrd.img-6.5.2-dirty, kernel_args="root=/dev/vda1 ro console=ttyS0,115200" \
    --os-variant debian10
\end{minted}
\caption{Build topology start command}
\label{lst:cmd_mk_topo_start}
\end{listing}

\begin{listing}[h]
\begin{minted}[linenos, frame=lines, framesep=2mm, breaklines]{json}
{
    "stub_range": "fd14:0::/32",
    "tran_range": "fd14:0::/32",
    "mgm_gw": "10.10.0.1/8",
    "stubs": [
        {
            "id": 10,
            "mask": 112,
            "area": 1,
            "router": 10,
            "clients": 5
        }, {
            "id": 11,
            "mask": 112,
            "area": 1,
            "router": 10,
            "clients": 5
        }, {
            "id": 20,
            "mask": 112,
            "area": 2,
            "router": 20,
            "clients": 5
        }, {
            "id": 21,
            "mask": 112,
            "area": 2,
            "router": 20,
            "clients": 5
        }
    ],
    "trans": [
        [[  1,   2], 0],  /* [[ <ROUTER ID>, <ROUTER ID>], <OSPF AREA>] */
        [[  1,  10], 1],
        [[  2,  20], 2],
    ],
    "area_boundary": [
        /* { <ROUTER ID>, <OSPF AREA>, <PUBLISHED NET MASK> } */
        { "id": 1, "area": 1, "mask": 96 },
        { "id": 2, "area": 2, "mask": 96 },
    ]
}
\end{minted}
    \caption[Network topology configuration file format]{
        Network topology configuration file format.
        The resulting topology is illustrated in \autoref{fig:ex_topo}
    }
    \label{lst:topo_cfg_format}
\end{listing}

\begin{listing}[h]
\begin{minted}[linenos, frame=lines, framesep=2mm, breaklines, fontsize=\small]{text}
Usage: mdc_send [OPTIONS...] [PEER ADDRESSES] [PEER PORT]
MEADcast sender

 Network:
  -a, --address=ipv6         Specify source address.
  -i, --interface=ifname     Specify interface to use.
  -p, --port=port            Specify source port.
      --tun-address=ipv6     Specify host route to send traffic to.
  -t, --tun=ifname           Specify name of the created TUN device.

 Discovery:
  -d, --delay=secs           Specify delay until initial discovery phase.
  -I, --interval=secs        Specify discovery interval.
  -T, --timeout=secs         Specify discovery timeout.
  -w, --wait                 If specified, sender waits for traffic before
                             starting discovery phase.

 Grouping:
      --max=max              Specify the maximal number of addresses per
                             MEADcast packet.
      --min-leaves=leaves    Routers with less leaves than `leaves` and less
                             routers than `routers` get removed from tree.
      --min-routers=routers  See `min-leaves`.
  -m, --merge=distance       Specify whether to merge leaves with distinct
                             parent routers under an common ancestor.
                             `distance` determines the range within which leaves
                             will be merged. Assigning `distance` a value of 0
                             disables this feature.
      --ok=ok                Specify whether a packet can be finished
                             prematurely if it contains an equal or greater
                             number of addresses than `ok`. Assigning `ok` a
                             value greater or equal than `max` disables this
                             feature.
  -s, --split                Specify whether leaves of same parent can be split
                             into multiple packets. If a router has more leaves
                             than the maximal number of addresses per packet,
                             they will be split regardless of this flag.

 Verbosity
      --print-groups         Prints grouping to stdout
      --print-tree           Prints topology tree to stdout

  -?, --help                 Give this help list
      --usage                Give a short usage message
  -V, --version              Print program version

Mandatory or optional arguments to long options are also mandatory or optional
for any corresponding short options.
\end{minted}
\caption{MEADcast sender help prompt}
\label{lst:send_help}
\end{listing}


\begin{listing}[h]
\begin{minted}[linenos, frame=lines, framesep=2mm, breaklines, fontsize=\small]{bash}
sudo ip6tables -A INPUT -i enp2s0 -j DROP
sudo ip6tables -A FORWARD -i enp2s0 -j DROP
\end{minted}
\caption{Simulate link failure between R1 and R3 by dropping all packets on R3}
\label{lst:r3_link_failure}
\end{listing}


\begin{listing}[h]
\begin{minted}[linenos, frame=lines, framesep=2mm, breaklines, fontsize=\small]{bash}
# enable
sudo ip6tables -A INPUT -i enp2s0 -m ipv6header --header ipv6-route --soft -j DROP
sudo ip6tables -A FORWARD -i enp2s0 -m ipv6header --header ipv6-route --soft -j DROP

# disable
sudo ip6tables -F
\end{minted}
\caption{Enable/Disable firewall on R3}
\label{lst:r3_fw}
\end{listing}
% chapter Code (end)

\begin{listing}[h]
\begin{minted}[linenos, frame=lines, framesep=2mm, breaklines, fontsize=\small]{bash}
HOSTS="/home/rnp/experiments/ex1_stream/util/s1_rx"; \
    ~/rx $HOSTS start && \
    ssh -i ~/id_rsa_vms user@10.10.30.1 'time parallel -v -a s1_tx -j $(cat s1_tx | wc -l) iperf -c {} -u -V -e --no-udp-fin -b 5000k -t 45' && \
    sleep 3 && \
    ~/rx $HOSTS stop ~/experiments/ex1_stream/s1/res/uni/5_45s/  > /dev/null
\end{minted}
    \caption[Start command IP Unicast measurement]{Start command IP Unicast measurement.
        The \texttt{"rx start"} command connects to each address in \texttt{"\$HOSTS"} and starts Iperf in server mode.
        The sender 10.10.30.1 transmits 5 Mbit/s for 45 seconds to all addresses in \texttt{"s1\_tx"}.
        The \texttt{"rx stop"} command connects to each receiver, stops Iperf and copies the result to the destination directory.
    }
\label{lst:uni_start_cmd}
\end{listing}

\begin{listing}[h]
\begin{minted}[linenos, frame=lines, framesep=2mm, breaklines, fontsize=\small]{bash}
HOSTS="/home/rnp/experiments/ex1_stream/util/s5_rx"; \
    ~/rx $HOSTS start multi fd14::3:30:1  && \
    ssh -i ~/id_rsa_vms user@10.10.30.1 'time iperf -c [ff3e::4321:1234]%enp2s0 -T 32 -u -V -e --no-udp-fin -b 2500k -t 45' && \
    sleep 3 && \
    ~/rx $HOSTS stop ~/experiments/ex1_stream/s5/res/multi/3_45s/ > /dev/null
\end{minted}
    \caption[Start command IP Multicast measurement]{Start command IP Muticast measurement.
        The \texttt{"rx"} script connects to each address in \texttt{"\$HOSTS"} and starts Iperf in server mode listening in \gls{ssm} to \texttt{"fd14::3:30:1"}.
        The sender 10.10.30.1 transmits 2.5 Mbit/s for 45 seconds via IP Multicast.
        The \texttt{"rx stop"} command connects to each receiver, stops Iperf and copies the result to the destination directory.
    }
\label{lst:multi_start_cmd}
\end{listing}

\begin{listing}[h]
\begin{minted}[linenos, frame=lines, framesep=2mm, breaklines, fontsize=\small]{bash}
HOSTS="/home/rnp/experiments/ex1_stream/util/s5_rx"; \
    ~/rx $HOSTS start && \
    ssh -i ~/id_rsa_vms user@10.10.30.1 "time iperf -c fd15::1 -u -V -e --no-udp-fin -b 1000k -t 45 -l $(echo '1484 - 20 * 16 - 52' | bc)" && \
    sleep 3 && \
    ~/rx $HOSTS stop ~/experiments/ex1_stream/s5/res/mead/1_45s/merge/2_20/ > /dev/null
\end{minted}
    \caption[Start command MEADcast measurement]{Start command \gls{mead} measurement.
        The \texttt{"rx"} script connects to each address in \texttt{"\$HOSTS"} and starts Iperf in server mode.
        The sender 10.10.30.1 transmits 1 Mbit/s for 45 seconds to \texttt{"fd15::1"}, which is usually the address of the TUN interface.
        The \texttt{"rx stop"} command connects to each receiver, stops Iperf and copies the result to the destination directory.
    }
\label{lst:mead_start_cmd}
\end{listing}

\begin{listing}
\begin{minted}[linenos, frame=lines, framesep=2mm, breaklines, fontsize=\small]{bash}
# enable / disable MEADcast
echo 1 > /proc/sys/net/ipv6/meadcast/enable

# set minimum destination count for premature M2U
echo 3 > /proc/sys/net/ipv6/meadcast/min_dsts
\end{minted}
    \caption[MEADcast router configuration]{MEADcast router configuration }
\label{lst:rt_cfg}
\end{listing}
